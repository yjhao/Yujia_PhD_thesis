\documentclass{umthesis}
\usepackage{graphicx}
\usepackage{psfrag}
\usepackage{epstopdf}
\usepackage{amsmath}
\graphicspath{{./figures/}}

\DeclareMathOperator*{\argmin}{arg\,min} % 

%%------------------------------------------------------
\begin{document}

%%
%% You must fill in all of these appropriately
\title{Wind Farm Wake Modeling and analysis of wake impacts in a wind farm}
\author{Yujia Hao}
\date{April 2016} % The date you'll actually graduate -- must be
                     % February, May, or September
\copyrightyear{2016}
\bachelors{B.Sc.}{Chongqing University}
\committeechair{Matthew Lackner}
\firstreader{Blair Perot}
\secondreader{Sanjay Arwade}
\departmentchair{Sundar Krishnamurty} % Uses "Department Chair" as the title.
\departmentname{Mechanical Engineering}


%%
%% These lines produce the title, copyright, and signature pages.
%% They are Mandatory; except that you could leave out the copyright page
%% if you were preparing an M.S. thesis instead of a PhD dissertation.
\frontmatter
\maketitle
\copyrightpage     %% not required for an M.S. thesis
\signaturepage

\begin{dedication}              % Dedication page
  \begin{center}
    \emph{Dedication page.}
  \end{center}
\end{dedication}

\chapter{Acknowledgments}             % Acknowledgements page
Acknowledgements page



%%
%% Abstract is MANDATORY. -- Except for MS theses
\begin{abstract}                % Abstract
More and more wind turbines have been grouped in the same location during the last decades to take the advantage of profitable wind resources and reduced maintenance cost. However wind turbines located in a wind farm are subject to a wind field that is substantially modified compared to the ambient wind field due to wake effects. The wake results in a reduced power production, increased load variation on the waked turbine, and reduced wake farm efficiency. Therefore the wake has long been an important concern for the wind farm installation, maintenance, and control. Thus a wake simulation tool is required. Due to the temporal and spatial variability of wind speed, direction, turbulence, and atmospheric stability, it becomes very challenging to accurately estimate the wake profile and the power losses due to the wake. The current tools that are used to model the wake are either not accurate enough or require too much computation time. This research creates and develops superior approaches to the traditional wind farm wake analysis tool. Three major contributions are presented: creation and utilization of a wind farm wake model that accurately and efficiently addresses the wake effects in an arbitrary wind farm with arbitrary inflow condition, new versatile statistical and efficient approaches for the meandered wake center modeling, and new technical approaches to model the dynamic wake effects of both onshore and floating wind turbines that could be further developed for control needs. These new modeling approaches and technical strategies are unified into a comprehensive Wind Farm Modeling Program (WFMP). With the incorporation of FAST,  WFMP provides a unified, flexible, and efficient approach for wind farm efficiency estimation and turbine loads assessment. In addition it enables several other analysis, such as mooring dynamics analysis and hydro-elastic analysis of waked offshore wind turbines, both of which were not able to be performed until WFMP is created. WFMP can drastically improve wind farm design, modeling, and control.

\end{abstract}

%%
%% Preface goes here...would be just like Acknowledgements -- optional
%% \chapter{Preface}
%% ...


%%
%% Table of contents is mandatory, lists of tables and figures are
%% mandatory if you have any tables or figures; must be in this order.
\tableofcontents                % Table of contents
\listoftables                   % List of Tables
\listoffigures                  % List of Figures

%%
%% Dedication is optional -- but this is how you create it
%%\begin{dedication}              % Dedication page
%%  \begin{center}
%%    \emph{To those little lost sheep.}
%%  \end{center}
%%\end{dedication}

%%----------------------------------------------------------------------------------------------------------------------------%%%%
% main matter
%%----------------------------------------------------------------------------------------------------------------------------%%%%

\mainmatter   %% <-- This line is mandatory

\chapter{Thesis overview and contributions}
A wind turbine is used to extract the kinetic energy from the wind and convert it to electrical energy. In modern wind energy development, multiple wind turbines are grouped in the same location to form a wind farm. Because profitable wind resources are limited to distinct geographical areas, the introduction of multiple turbines into these areas increases the total energy produced, and the concentration of repair and maintenance equipment reduces the total cost. Due to the energy extracted by the rotor, however, the wind field behind the wind turbine has a lower mean velocity and higher turbulence intensity compared with the ambient wind. In most cases, the wake reaches the downwind turbines before it fully recovers to ambient wind conditions. This results in reduced power production and increased load variation at the downwind turbine. Wind turbine wakes have long been an important concern for wind farm design, operation, and control.

Wake modeling in a wind farm is very challenging, not only because the single wake evolution and recovery are related to several interdependent atmospheric conditions, such as the ambient wind speed, turbulence intensity, wind shear, and stratification; but also because in a wind farm, the wakes from several different turbines merge laterally and vertically.

This thesis is motivated by the significant wind farm wake effects and the rapid growth of wind farm deployment, requiring a model that can accurately and efficiently calculate the wake impacts in a wind farm. These capabilities are the essential prerequisite to design, operate, and control a wind farm. By reviewing the drawbacks of current wake models, a tool that yields satisfactory results and maintains acceptably low computation cost is needed.

In this research, three major contributions are presented: (i) creation and utilization of a wind farm wake model that accurately and efficiently addresses the wake effects in an arbitrary wind farm with arbitrary inflow condition; (ii) the development of a new, versatile, statistical and efficient approach for the meandered wake center modeling; and (iii) the development of a new approach to model the dynamic wake effects of both onshore and floating wind turbines that may be further developed for integrated wind farm control studies.

New wake modeling approaches for a wind farm are created and developed. In this research, the single wake model is based on the Dynamic Wake Meandering (DWM) model. Novel algorithms are developed to systematically model the wake effects and thus the power production of an entire wind farm with arbitrary wind turbine layout and wind condition. This wind farm model is implemented into an open-source framework and integrated with an aero-elastic code, so it can be applied to model waked turbine power and loads. The model agrees well with the results of the Simulator for Wind Farm Applications (SOWFA), a high fidelity Computational Fluid Dynamics (CFD) Large Eddy Simulation (LES) model, and experimentally-obtained field data, while maintaining an acceptably low computational cost. With this model, for a wind farm with arbitrary turbine layout and arbitrary inflow, both the turbine power and loads could be efficiently and simultaneously calculted.  

New statistical approaches based on a Markov Chain are developed to model the meandered wake center locations. Compared with the traditional low-pass filter model, this new model yields accurate results, but is very efficient in both computation resource and computational time, and is versatile since it is solely based on a probability distribution. With this developed model, faster wake modeling is achieved, which is the prerequisite for larger design, control, and optimization studies.

Currently a tool that can examine the dynamic control of turbines in a wind farm is lacking. To address this issue, a novel approach to model the dynamic wake effects on the waked turbines is developed. This model is capable of modeling the dynamic transient wake impacts on the downwind waked turbine using a moving average window approach. The model could be further developed for wind farm dynamic control applications. In addition, this wind farm wake model is modified to simulate the dynamic wake of an offshore floating wind turbine and therefore the downwind turbine's power and loads.

These new modeling approaches and strategies are unified into a comprehensive Wind Farm Modeling Program (WFMP). WFMP is incorporated into the NWTC design codes, a state-of-the-art computer aided simulator capable of simulating horizontal-axis wind turbines. FAST is widely used in industry, making this tool easy to access and giving it a tremendous potential to be modified and developed for many other new analyses. With the integration of FAST,  WFMP provides a unified, flexible, and efficient approach for wind farm efficiency estimation and turbine loads assessment. In addition it enables several other analysis, such as mooring dynamics analysis and hydro-elastic analysis of waked offshore wind turbines, both of which were not able to be performed until WFMP was created. WFMP can drastically improve wind farm design, modeling, and control.

Chapter 2 provides an introduction to wind farm wakes and the current state of wind farm wake modeling. Chapter 3 presents the single wake DWM model in detail. Chapter 4 and 5 discuss the approaches to create the wind farm wake model and incorporates this wind farm model into the NWTC design codes as the Wind Farm Modeling Program (WFMP). In Chapter 6, some model comparison and validation results between WFMP and the high fidelity LES as well as the experimentally-obtained field data are shown. Chapter 7 provides new approaches to model the meandered wake center locations. Chapter 8 discusses the novel approach for capturing the wake transient state and modeling the dynamic wake impacts on the waked turbines. In Chapter 9 WFMP shows the capability of modeling the dynamic wakes of floating wind turbines.  Finally, Chapter 10 summarizes the overall conclusions of this study, and proposes future avenues of investigation.


\chapter{An introduction to wind farm wakes and wake modeling}
This chapter provides a brief background section on turbine wakes, the impact of the turbine wakes on the waked turbine in a wind farm, and the main factors that affect the turbine wake. In addition, previous work on wake modeling is presented. Many of the methodologies and techniques in this section are applied to this dissertation's research.


\section{Wind energy in the world}
Wind power, as an alternative to fossil fuels, is plentiful, renewable, widely distributed and produces no greenhouse gas emissions during operation. It plays an important role in today's energy supply and is expected to maintain rapid expansion in the years to come. According to the 2015 report from the World Wind Energy Association (WWEA) \cite{WWEA_a}, some facts are listed below:

\begin{itemize}
  \item The worldwide wind capacity reached 435 GW, out of which 63,690 MW were added in 2015.
  \item Among the top 15 markets, China, United States, and Germany are the top 3 countries, with total capacity of 148 GW, 74 GW, and 45 GW respectively.
  \item In 2015, the global growth rate of 17.2\% is higher than in 2014 (16.4\%). 
  \item WWEA expects by the end of year 2020, at least 700 GW installed globally. A global wind capacity of 2,000 GW is possible by the year 2030.
\end{itemize}

Wind power in the United States has been expanding rapidly over the last several years. As of the end of 2015, the capacity is 74,347 MW, and is exceeded only by China \cite{WWEA_a}, with a total of 39 states and Puerto Rico now having installed at least some utility-scale wind power \cite{AWEA}. For the years of 2014 and 2015, the electricity produced from wind power in the United States amounted to 19.69 TWh, or 6.52\% of all generated electrical energy, a 4.12\% increase from the year 2013 \cite{EPM}. The U.S. Department of Energy's 2008 report ``20\% Wind Energy by 2030'' envisioned that wind power could supply 20\% of all U.S. electricity, which included a contribution of 4\% to the nation's total electricity from offshore wind power \cite{thirty_by_twenty}. 

Besides the fact that wind power is renewable and emissionless, the main reason for the rapid growth of wind power is the more competitive price. According to a report by U.S. Department of Energy, in the year 2018 the total system levelized cost of wind power (onshore) is projected to be \$86.6/MWh, only higher than natural gas \cite{energy_cost}. In contrast, the total system levelized costs of conventional coal and advanced nuclear are projected to be \$100.1/MWh and \$108.4/MWh respectively.


\section{Wind turbine wake deficits}\label{sec:wake_review}
There are several advantages to grouping wind turbines in a wind farm. Profitable wind resources are limited to distinct geographical areas. The introduction of multiple turbines into these areas increases the total energy produced. The concentration of repair and maintenance equipment and spare parts reduces cost. In wind farms of more than about 10 or 20 turbines, dedicated maintenance personnel can be hired resulting in a reduced labor costs per turbine and financial savings to the wind turbine owners \cite{WEE}.

However, wind turbines located in a wind farm are subject to the wake - a wind field that is substantially modified compared to the ambient wind field. When the wind flows past a wind turbine, part of the wind's kinetic energy is extracted by the wind turbine and is converted to mechanical then electrical energy. The main characteristics of the wake are the lower mean wind speed and the larger fluctuations. The lower mean wind speed is due to the extraction of the kinetic energy and the larger turbulent fluctuations are due to the disturbance by the rotor and the mixing with the free stream flow. The power production of a downwind turbine is reduced as a result of the lower mean wind speed, and the blade fatigue loads are increased due to the enhanced turbulence, which may result in a shortened rotor lifetime. From the work of Barthelmie \cite{Meteorological_controls, Ten_years, Barthelmie_Evaluation}, field data shows that the reduction of annual power production is on the order of 5\% to 30\% and also depends on several other factors (explained later in this section).


Figure \ref{fig:actuator_disk} is a schematic of how the wind turbine extracts power from the inflow wind by modeling the rotor as an actuator disk. When the ambient flow approaches the rotor, the static pressure of the ambient flow increases from $P_\infty$ to $P^+$$_d$ just in front of the rotor plane, and then drops suddenly to $P^-$$_d$ behind the rotor. The pressure drop is associated with the force exerted on the rotor. The pressure gradually recovers to the freestream static pressure value $P_\infty$ in the wake. Since the turbine extracts energy from the wind, the wind velocity $U_\infty$ is reduced as the fluid element approaches and then passes through the rotor plane, and due to the fact that the static pressure increases downstream of the turbine, the velocity behind the turbine continues to decelerate and the wake region expands. The location in the wake with the lowest velocity coincides with the location where the pressure has recovered to the ambient pressure behind the rotor. This location sometimes is referred to as the end of the near wake region as shown in Figure \ref{fig:actuator_disk}.

\begin{figure}
  \centering
  \includegraphics[scale=0.32]{actuator_disk}
  \caption{An energy extracting actuator disc and stream-tube.}\label{fig:actuator_disk}
\end{figure}

Figure \ref{fig:wake} is a schematic of the wake velocity profile. The wake is mainly characterized by two regions in the axial direction - the near wake and far wake. Radially, a region of large velocity gradients is generated at the outer part of the wake deficit, where the low speed wake mixes with the high speed ambient wind. This region is called the \textit{wake shear layer}, and the enhanced shear in this region is the key driver of the wake recovery, and is transmitted into the inner wake region as the wake progresses downstream.

\begin{figure}
  \centering
  \includegraphics[scale=0.24]{turbine_wake}
  \caption{Schematic of wake profile between turbines in a row.}\label{fig:wake}
\end{figure}


In the near wake region, approximately 1-2 rotor diameters (D) downstream of the turbine, the local pressure field at the rotor is important for the development of the wake deficit. Further downstream, the shear layer continues to expand and finally reaches the wake center at $2-6D$ behind the turbine \cite{Aerodynamics}. For the far wake region, local variations at the rotor are diffused and higher momentum air is drawn into the wake volume as the wake mixes with the ambient wind around it, thus dissipating the wake. At a certain downstream location that depends on the ambient conditions, the wake attains a near-Gaussian shape that is axisymmetric and self-similar, and the crucial factor describing the wake evolution becomes the turbulence intensity and the turbulent mixing caused by the velocity gradients of the wake itself \cite{Meteorological_controls}. Thus it is possible to use parameters such as wind speed, rotor radius, and turbulence intensity to describe the wake distribution.

Due to the mixing between the wake and ambient flow, the wake expands as it travels downstream. In a wind farm, a downstream turbine may not see the ambient flow any more, but instead may experience the wake generated by the upwind turbine. Another crucial characteristic of wake is that the wake center meanders with the wake advection, which may cause the partial overlap of the wake on a downwind turbine as shown in Figure \ref{fig:wake}.

Wake meandering is the term used to describe the large-scale lateral or vertical movement of the entire wake. Wake meandering is important because it might considerably increase extreme loads and fatigue loads on the downstream turbines in wind farms, due to the fact that the meandering causes the wake to be swept laterally and vertically in and out of the rotor plane of downstream turbines. The phenomenon of wake meandering has long been known empirically since 1980s \cite{Ainslie_flow}\cite{Ainslie_wake}, but lacked a satisfactory model to capture the movement of the wake. Larsen et al. proposed to model the wake meandering by modeling the wake deficit movement as a passive tracer that is driven by the large scale eddies in the turbulence field \cite{Larsen_meandering}. Recently wind tunnel experiments are frequently employed to investigate the phenomena of wake meandering \cite{Medici}. Aubrun\cite{Aubrun} and España et.al. \cite{España_Spatial}\cite{España_tunnel} conclude using experiments that wake meandering is governed by the incoming turbulence and is correlated to the large eddies, which matches the assumption that Larsen proposed. However, further research is needed to determine exactly which scales of the incoming turbulence affect the large scale wake movement. Current wind tunnel research by Muller \cite{Müller} suggests that the turbulent eddies larger than 2-3D govern the wake meandering formulation, which corresponds well to the cut-off frequency proposed by Larsen et al. \cite{Larsen_meandering}.




\section{Key parameters for wind turbine wake evolution}
As mentioned above, the wake evolution and recovery are affected by several factors that make wind turbine wake modeling a challenging task. Furthermore, many of these parameters are correlated to each other so it is difficult to isolate one parameter from the ambient condition during field measurements and investigations. The nature of turbulence, which is highly non-linear, unsteady and random also increases the difficulty to model the wind turbine wake. The key parameters that affect the wind turbine wake evolution and recovery are discussed in the following sections.


\subsection{Turbulence intensity}
The ambient turbulence intensity is the most important parameter for wake evolution and is defined by:
\begin{equation}
  I_u=\frac{\sigma}{\overline{u}}
\end{equation}
Where $\sigma$ is the standard deviation of the wind velocity in the average wind direction, and $\overline{u}$ is the magnitude of the average wind velocity. In principle the intensity is different for each wind direction, from which the definitions for $I_u$,$I_v$ and $I_w$ follow, which are the turbulence intensity in the axial, lateral, and vertical directions, respectively.

Figure \ref{fig:Wake_evolution} \cite{Troldborg_ACL} shows the wake evolution as iso-contours of vorticity from CFD simulations behind an NM80 wind turbine operating in a wind speed of 10m/s. In the top figure the ambient flow is laminar and in the bottom figure the ambient turbulence intensity is 9\% \cite{Troldborg_ACL}. It clearly demonstrates that higher turbulence intensity increases the mixing rate between the wake and ambient wind, thus increasing the wake recovery. Also the recent work by Barthelmie and Jensen \cite{Barthelmie_Evaluation}, Hansen et al. \cite{TI_impact} and Troldborg \cite{Troldborg_ACL} show that increased turbulence intensity causes faster wake deficit recovery. Furthermore, Barthelmie et al. quantitatively concludes that the wind farm efficiency increases 0.98\% to 1.40\% for every 1\% turbulence intensity increase based on the findings from the Horns Rev and Nysted wind farms \cite{Meteorological_controls}.

\begin{figure}
  \centering
  \includegraphics[scale=0.5]{Wake-evolution}
  \caption{Wake evolution as iso-contours of vorticity from CFD simulations.}\label{fig:Wake_evolution}
\end{figure}


\subsection{Atmospheric stability}
Atmospheric stability indicates the degree of stratification of the atmosphere and how much mixing occurs between the air of these different layers. Since the wake is highly impacted by the flow mixing between the wake and the ambient wind, atmospheric stability is a crucial factor for wake evolution.

To classify the atmospheric stability, the Monin-Obukhov length scale is introduced and is defined as \cite{Stull}:
\begin{equation}
  L=\frac{-\overline{\theta_v}u_*^3}{\kappa g(\overline{w'\theta '_v)}_s }
\end{equation}

Where $u_*$ is the friction velocity, $\kappa$ is the von Karman constant and $g$ is the gravitational acceleration. $w'\theta '_v$ is the flux of virtual potential temperature, and the subscript $s$ denotes that this variable is obtained at surface level.

Based on the Monin-Obukhov length scale, the atmospheric stability can be classified using several different stability classes. According to Sathe et al. there is currently no firm criterion to define the limits of L to define the different stability classes \cite{Sathe}, they are only based on previous research experience, different stability classifications exist in the literature. In general, atmospheric stability can be classified as unstable, neutral and stable. However, classifications as very (un)stable or near-neutral (stable/unstable) are found in the literature as well \cite{Ten_years, Barthelmie_as, Sathe_sitedata, Sathe, Wharton_2, Wharton_1}.

It is clear that increased atmospheric mixing should tend to promote recovery of wind turbine wakes and thus increased wind farm efficiency. However, it is difficult to quantify the functional dependence of wake losses on atmospheric stability precisely, partially because parameters such as TI, wind speed and atmospheric stability are all correlated to each other. A number of studies using measurements (Barthelmie et al. \cite{barthelmie_modelling_2009}, Barthelmie et al. \cite{Barthelmie_wakes_2009}, Barthelmie et al. \cite{Ten_years} and Hansen et al. \cite{hansen_impact_2012}) and simulations (Larsen et al. \cite{Larsen_2009}, Churchfield et al. \cite{Churchfield_2012}, Lee et al. \cite{Lee_2014} and Lavely et al. \cite{Lavely_2011}) investigate the effect of atmospheric stability on the wake evolution and recovery.


\subsection{Wind farm effects}
Wake formation and recovery in a wind farm are subject to the placement of the wind turbines and the wind direction. Figure \ref{fig:wake_wind_farm} is a famous Horns Rev wind farm picture which was taken on February 12, 2008. In this picture, the unique meteorological conditions resulted in the wind turbines creating condensation (i.e. clouds) of the very humid air, thus making it possible to see the wakes and turbulence patterns behind the wind turbines. It is clearly seen that in a wind farm, not only does downstream axial wake merging occurs, but also lateral wake merging. Downstream axial wake merging occurs when several upwind turbine wakes reach a downwind turbine and a new wake is released from this downwind turbine; lateral wake merging occurs when the wakes from two parallel wind turbines interact. Wake merging limits wake recovery due to the fact that the momentum of the air surrounding the wake is lower than the ambient wind \cite{Barthelmie_2011_Meteorological}. Experimental tests also show that the enhanced small scale turbulence creates a wider wake deficit with a lower velocity gradient \cite{Barthelmie_2011_Meteorological, Méchali}. This explains why in the center of a larger wind farm, the turbine power production continues decreasing compared with the nearest upwind turbine, rather than recovering as a result of the enhanced turbulence intensity \cite{Hao}. This phenomenon is referred to as the deep array effect \cite{Andersen_2011, Barthelmie_Evaluation, Calaf_2010, Frandsen_2007, meyers_optimal_2012, Schlez_2009, Smith_2011}, which often happens in a large wind farm as shown in Figure \ref{fig:deep_array_effect}.

After a sufficient number of wind turbines, the inflow wind for the downwind turbine is considered to be in equilibrium, and is not able to further recover, which in turn means that the power production and the loads for the subsequent turbines are very similar. In this situation, most of the momentum transferred into the wake region from the ambient flow enters vertically, and thus is directly related to the atmospheric stability.

\begin{figure}
  \centering
  \includegraphics[scale=0.5]{wake_wind_farm}
  \caption{Wakes in Horns Rev wind farm}\label{fig:wake_wind_farm}
\end{figure}

\begin{figure}
  \centering
  \includegraphics[scale=0.5]{deep_array_effect}
  \caption{Normalized velocity field from a CFD simulation at Lillgrund wind farm.}\label{fig:deep_array_effect}
\end{figure}

Based on the wind direction, axial wake merging can be further separated into full wake merging and partial wake merging. Full wake merging is when the ambient flow is perfectly down the row, whereas partial wake merging occurs when there is an offset between the ambient wind direction and the turbine row, resulting in the downwind turbine suffering higher loads variation than a turbine in a full wake. It is found in some results that the 2nd turbine in a row experiences the lowest inflow wind speed and thus the lowest power production \cite{Schepers_2009}. Wind tunnel results show that the turbulence intensity and the turbine power production reach an equilibrium state after 3-4 turbines in a row \cite{Vermeulen_1982}. Both findings match the results by Hao et al. \cite{Hao}. However, for partial wake merging, it is more difficult to describe exactly how much of the wake impacts a downwind turbine. It is experimentally revealed that partial wake operation causes non-symmetrical wake emitting and the released wake direction is altered \cite{Troldborg_2011}. It is also concluded by Barthelmie \cite{Meteorological_controls} that for turbines in a partial wake operation state, the wake deficit at the $2^{nd}$ turbine is not the greatest, but the downstream turbines' power production continues to decrease.


\subsection{Turbine operation state}
It is typical to model the wind turbine rotor as an actuator disk (as shown in Figure \ref{fig:actuator_disk}) which induces velocity variations of the free stream velocity. The axial component of the induced velocity at the disk is given by $-aU_\infty$, where $a$ is called the \textit{axial induction factor} and is defined as the fractional decrease in the wind speed at the rotor disk from the free stream wind speed. Thus, the velocity at the rotor plane is given by:
\begin{equation}
 U_d=U_\infty(1-a)
\end{equation}
Based on momentum theory, the waked velocity behind the rotor plane $U_w$ can be derived as:
\begin{equation}
 U_w=(1-2a)U_\infty
\end{equation}
Thus the induction factor directly affects the boundary condition of the wake recovery. On the other hand, the induction factor is a nonlinear function of the tip speed ratio, and the blade pitch angle, both of which depend on the local instantaneous wind speed. A wind turbine mainly operates in two different regions. When the wind speed is below a certain value that corresponds to the rated power of the turbine, the turbine attempts to maximize the capture of the kinetic energy in the wind by operating at the optimum induction factor, which is $1/3$ based on Betz theory. This is achieved either by changing the pitch angle, or using torque control to adjust the rotor speed. At wind speeds above the rated value, to reduce the thrust force on the blades while maintaining the rated power generation, the turbine attempts to reduce the induction factor by stalling the blades, pitching the blades, or some combinations of both. Sometimes, it is useful to link the thrust coefficient ($C_T$) and the wake formulation and recovery, such that the higher the $C_T$, the wider the wake expands.



\section{Wake modeling approaches}
Depending on the layout and the wind conditions of a wind farm, the power loss of a downstream turbine can easily reach 40\% in full-wake conditions. When averaged over different wind directions, losses of approximately 8\% are observed for onshore farms and 12\% for offshore farms \cite{Barthelmie_2008}. Thus it is important to accurately model and predict the wind turbine wakes and the wake interactions in terms of the effects on the wind turbine performance. In practice, there are a range of wake modeling methods deployed, which can be categorized based on the fidelity level.


\subsection{High Fidelity Models}
High fidelity models include Large Eddy Simulation (LES) \cite{Pope} and Reynolds Averaged Navier Stokes (RANS) method. To obtain a reliable result, a high level of detail is required in these models and few assumptions are made. In general, the near-wake vortex shedding from the blades, and the initial interaction between atmospheric turbulence and the wake are of interest within the high fidelity models. The blades are modeled as solid boundaries in the CFD simulation. The advantage of this approach is that very few assumptions need to be made and the three dimensional pressure and force distribution can be directly reflected in the simulation domain, and as a consequence, the wake evolution especially in the near wake is accurately modeled. However, the drawback is also clear, as a very fine mesh is needed to resolve the boundary layer as well as the blade geometry, which greatly increases the calculation cost. Recently the actuator line (AL) model is used to represent the blade geometry as three rotating lines to capture the influence of the blades on the incoming ambient flows \cite{Sørensen_2002}. Though the AL model is (at least) an order of magnitude less computationally expensive compared to full rotor representations, these approaches are still very expensive compared to other models and are mainly research tools used to conduct numerical validation for low-fidelity model development.

Among these high fidelity models, recently the National Renewable Energy Laboratory (NREL) developed SOWFA \cite{SOWFA}, a computational fluid dynamics solver based on OpenFOAM (Open-source Field Operations and Manipulations) libraries \cite{openfoam} coupled with FAST (the NREL’s open-source wind turbine simulator \cite{FAST}) that allows users to investigate wind turbine performance under various atmospheric conditions. SOWFA has been validated with various field data and a good agreement is shown. Therefore the turbine performance in a wind farm under the wake influence can be accurately simulated and an individual turbine can be controlled with respect to the instantaneous altered inflow wind. SOWFA requires large amounts of computing power and is typically run on a supercomputer.

\subsection{Medium Fidelity Models}
Medium fidelity models are based on the Navier Stokes (N-S) equations and include most of the important wake physics, but less important features are simplified for the application. For instance, the wind turbine may be represented by using the actuator disc method and the time averaged physics \cite{Mikkelsen_2003}. This method is often used for investigating the far-wake effects where an acceptably accurate result is  yielded, whereas the near-wake effects and the vortex structure are not included in this model. Besides the methods that deploy the N-S equation to model the wake, free wake models based on vortex segments are also commonly used. This method is based on modeling the vortex structures of the wake and finding the inviscid flow field by using the Biot-Savart integral, and provides a cost-effective way to obtain the rotor induction and the near-wake physics, though it can hardly be used to simulate the far-wake flow field since the wake recovery effect is not included in this model. In order to model the far-wake while using the Biot-Savart integral, a viscous model needs to be coupled to include the turbulent mixing effects. A more thorough review of the vortex wake modeling is given by Sorensen\cite{Aerodynamic_aspects} and Vermeer \cite{Vermeer_aerodynamics}.


\subsection{Low Fidelity Models}
Fast tools are needed for engineering design and industrial applications. These models are based on reduced order physics to maintain acceptable accuracy and affordable calculation cost. Due to the simplified physics, most of the engineering models are not applicable for near-wake effects. Currently there are a number of engineering models that have been developed. Within all of the low fidelity models, the Jensen/ Katic model is the most popular one that may be used for micrositing and wind farm output predictions.

The Jenson/Katic model is based on momentum conservation of the deficit in the wake caused by the wind turbine. The wake is assumed to have a uniform wind speed in the cross flow direction, and the wake expands linearly and radially with downstream distances according to the wake decay constant, $k$, an empirical tuning parameter, which is claimed to be a function of the incoming wind turbulence intensity and atmospheric stability \cite{Duckworth_2008}. The wake deficit is formulated as:

\begin{equation}
  a=\frac{1}{2} (1-\sqrt{1-C_T})
\end{equation}

\begin{equation}
  1-\frac{U_x}{U_0}=\frac{(1-\sqrt{1-C_T})}{(1+2k\frac{X}{D})^2}
\end{equation}

The initial non-dimensional velocity deficit (the axial induction factor), $a$, is assumed to be a function of the turbine thrust coefficient. The initial free stream velocity is $U_0$ and the turbine diameter is $D$. The velocity in the wake at a distance $X$ downstream of the rotor is $U_X$ with a wake diameter of $D_X$.

This model assumes that the kinetic energy deficit of interacting wakes is equal to the sum of the energy deficit of the individual wakes (indicated by subscripts $1$ and $2$). Thus the velocity deficit at the intersection of two wakes is:

\begin{equation}
  (1-\frac{U_X}{U_0})^2=(1-\frac{U_X,_1}{U_0})^2+(1-\frac{U_X,_2}{U_0})^2
\end{equation}

Besides Jenson/Katic, other low fidelity models include the WASP (the Wind Atlas Analysis and Application Program), and the Frandsen model. WASP is a software package developed by Riso-DTU to model the wind resources for wind turbine micro-siting \cite{WASP}, and this model is based on linearized N-S equation. The Frandsen model is used to simulate the turbine's operation state with waked inflow wind \cite{Frandsen_2007}, by modeling the increased inflow effective turbulence intensity based on parametric scaling. This model currently is the IEC 61400-1 industry standard for increased turbine loads with waked inflow.


\section{Simulation tools}
This research mainly uses the computer-aided engineering (CAE) tools developed at the National Renewable Energy Laboratory (NREL), including FAST, AeroDyn, and TurbSim. Those programs all have been verified and validated against experiments, and provide reasonable accuracy at low computational cost. In this work, the wind farm model is implemented in FAST, reading the wind files generated by TurbSim and calculating the loads using AeroDyn.


\subsection{FAST and AeroDyn}
The FAST (Fatigue, Aerodynamic, Structures, and Turbulence) code is a comprehensive time-domain fully-coupled aero-hydro-servo-elastic simulator capable of simulating horizontal-axis wind turbines (HAWTs). This program is open source and Fortran based. It relies on a few modules to run.  Among these modules, AeroDyn is the aerodynamic module, which uses the blade-element momentum (BEM) theory to predict the blade loads, with tip / hub loss and dynamic stall corrections. It requires information on the status of a wind turbine from the dynamics analysis routine and returns the aerodynamic loads for each blade element to the dynamics routines \cite{FAST}.


\subsection{TurbSim}
To generate a wind file with a specific wind speed and turbulence intensity, TurbSim is used. TurbSim is a stochastic, full-field, turbulent-wind simulator. It uses a statistical model to numerically simulate the time series of three-component wind-speed vectors at points in a two-dimensional vertical rectangular grid that is fixed in space \cite{Turbsim}, and uses the inverse fast Fourier transform to introduce the turbulence and spatial coherence in the generated wind field \cite{Shinozuka}.


\subsection{Turbine models}
FAST requires the blade, rotor, and tower properties of the turbine to be modeled to run the simulation. However acquiring those details of a real modern turbine is difficult since this information is proprietary to the turbine manufactures. Thus in this work, a representative 5 MW wind turbine model developed by NREL \cite{NREL5MW}, a generic Vestas V80 model, and a generic Vestas V90 model \cite{Churchfield_turbine} are used. The NREL 5 MW wind turbine is based on the Repower 5MW machine. The V80 and the V90 model are obtained by producing a design with predicted performance that matches the manufacturer’s power curve as closely as possible based on the publicly available information.



\chapter{The Single wake model} \label{chap:DWM_detail}
The single wake model that is utilized in the wind farm model is based on the DWM model. This chapter introduces the technical details of the single wake model. Section \ref{sec:DWM_overview} presents a general outline of the DWM model. Section \ref{sec:wake_eqn} discusses the DWM wake deficit model governing equations. Section \ref{sec:wake_BC} presents how the wake deficit inlet boundary condition is determined. Section \ref{sec:numerical solution} discusses the numerical N-S equation solution applied in this dissertation. And last, the wake meandering sub-model is covered in section \ref{sec:wake_meandering}.

\section{Dynamic wake meandering model}\label{sec:DWM_overview}
Wind turbines located in a wind farm are subject to a wind field that is substantially modified compared to the ambient wind field. The significant change of the wind field due to the wakes of upstream wind turbines has an impact on the downwind waked turbine - not only on the power production but also on the mechanical loads. The evolution and recovery of the wake depend on several parameters relating to the ambient wind field, including the mean wind speed, turbulence intensity, turbulent length scale, wind shear and stratification. Moreover the fact that these aforementioned ambient parameters are interdependent makes the wake modeling more challenging.

A \textit{wake} is characterized by a mean wind decrease and turbulence increase behind a turbine. Also the wake center moves both laterally and vertically when it marches downstream; this stochastic pattern is called \textit{wake meandering}. Thus the resulting wind field may be considered as a turbulence field with a significantly increased turbulence intensity and substantially modified turbulence structure. The enhanced turbulence intensity contains the contribution from the generated turbulence by the wake itself, and the large scale apparent wake center movement.

The outline of the DWM model used in this study is based on and developed from the work presented by Larsen et al. \cite{Larsen_wake}. It consists of two sub-models - a model to estimates the quasi-steady wake deficit and a stochastic model of the downstream wake meandering process, as shown in Figure \ref{fig:DWM_overview} \cite{Keck_mixing}.

\begin{figure}
  \centering
  \includegraphics[scale=0.5]{DWM_overview}
  \caption{Overview of the DWM and its sub-models.}\label{fig:DWM_overview}
\end{figure}

The quasi-steady wake deficit is the wake deficit formulated in the moving (meandering) frame of reference, and the quasi-steady wake deficit model applies the two-dimensional eddy viscosity model \cite{Keck_mixing}. The wake deficit model includes the strain-rate contribution from the atmospheric boundary layer shear in the wake deficit calculation and the wake expansion as a function of downstream transport distance caused by turbulent diffusion and the rotor pressure field \cite{Keck_two}. The wake meandering model describes the stochastic downstream wake transport driven by large scale turbulent structures and is based on the model proposed by Larsen et al. \cite{Larsen_meandering}.

To estimate the turbine loads with a waked inflow, the turbulence intensity and turbulence structure are required, which are all captured by the DWM model. The DWM model is distinguished from other wake models by its ability to capture the time-dependent physics present in wind turbine wakes. As a consequence, both wind turbine loads and power production can be predicted at the same time. The DWM model solves the Navier-Stokes equations in a simplified, but acceptable manner, to simulate the quasi-steady wake deficit. It also uses large-scale turbulence to drive the wake meandering phenomenon. As a consequence, the DWM model may be used for the optimization of wind farm topology, individual turbine control, and wind farm global control.

For both of the sub-models, the 3-D space behind the wind turbines is divided into multiple adjacent 2-D crossing planes that are perpendicular to the axial wind direction. At each crossing plane, the 2-D plane is further discretized to a 2-D rectangular grid, where at each point a wake velocity is calculated by the wake deficit model. The meandered wake center coordinates are calculated at each crossing plane. During most of the modeling process, the two models are working separately.

The inputs of the DWM model are:
\begin{enumerate}
  \item The turbine induction factor, or the induced wind speed at the rotor plane.
  \item The ambient wind speed, or the characteristic wind speed at the rotor plane.
  \item The ambient turbulence intensity, and the enhanced turbulence intensity if the investigated turbine sees waked flow instead of the freestream.
\end{enumerate}

The outputs of the DWM model are:
\begin{enumerate}
  \item The 3-D wake field behind the investigated turbines.
  \item The time series of meandering wake center position in the 3-D field.
  \item The added turbulence intensity due to the wake formulation and the wake meandering on the downwind turbines if there are any.
\end{enumerate}

The sub-models of the DWM are treated separately in the following sections.


\section{Wake deficit sub-model governing equations}\label{sec:wake_eqn}
The wake deficit model included in the DWM model is inspired by the work of Ainslie \cite{Ainslie_flow}, who proposed the use of a thin-shear-layer approximation of the Navier-Stokes equation to model the wake behind a wind turbine by assuming the wake profile is axisymmetric. As mentioned before, because the mean pressure gradients are small relative to the turbulent mixing of the wake deficit in the far wake region, one can disregard the pressure term in the N-S equations, and assume that gradients of the quantities in the radial direction (denoted by $r$) are much larger than those in the axial direction (denoted by $x$). These modifications lead to the thin-shear-layer approximation of the Navier-Stokes equation. It is further assumed that the system is at a steady state, and so the governing axial momentum equation is given by:
\begin{equation}\label{eq:n_s}
  U\frac{\partial U}{\partial x}+V\frac{\partial U}{\partial r}=-(\frac{1}{r})\frac{\partial}{\partial r}(r\overline{uv})
\end{equation}
Where $U$ and $V$ denote the mean velocity in the axial and radial directions respectively, $u$ and $v$ denote the respective fluctuating velocity components in these directions, and an upper bar denotes averaging.

By introducing the eddy viscosity concept, the Reynolds stress $\overline{uv}$ term can be expressed as:
\begin{equation}\label{eq:nu_t}
  \overline{uv}=\nu_t\frac{\partial U}{\partial r}
\end{equation}
Where the $\nu_t$ term is the viscosity to be modeled. Combining Eq. \ref{eq:n_s} and Eq. \ref{eq:nu_t} yields the reformulated governing equation as:
\begin{equation}\label{eq:x_momentum_1}
  U\frac{\partial U}{\partial x}+V\frac{\partial U}{\partial r}=(\frac{\nu_t}{r})(\frac{\partial}{\partial r})(r\frac{\partial U}{\partial r})
\end{equation}
And the continuity equation is given by:
\begin{equation}\label{eq:continuity_1}
  \frac{1}{r}\frac{\partial}{\partial r}(rV)+\frac{\partial U}{\partial x}=0
\end{equation}

The viscosity in the wake field is considered to be dependent on both the ambient turbulence and the turbulence generated by the wake itself. Based on the definition of the eddy viscosity, the characteristic viscosity is given by \cite{Pope}:
\begin{equation}\label{eq:l*u*}
  \nu_t=l^*u^*
\end{equation}
Where $l^*$ and $u^*$ are the characteristic length scale and velocity scale respectively.

According to the classical formulation of the mixing length model, the local turbulent velocity scale is based on the local strain rate and the turbulence length scale \cite{Pope}:

\begin{equation}\label{eq:u*}
  u^*=l^*|\frac{\partial u}{\partial r}|
\end{equation}

Thus by combining Eq. \ref{eq:l*u*} and Eq. \ref{eq:u*}, the turbulent viscosity is given by:

\begin{equation}
  \nu_{t,shear}=l^{*2}|\frac{\partial u}{\partial r}|
\end{equation}

It has been shown to be more accurate if the viscosity distribution is dependent on the wake formulation both in the axial direction and the radial direction, rather than just dependent on the axial direction. Thus the modeled viscosity model equation is given by:

\begin{equation}
  \nu_{t,shear}=F_1k_1TI_{amb}+F_2k_2l^{*2}|\frac{\partial u}{\partial r}|
\end{equation}

Where $TI_{amb}$ is the ambient turbulence intensity, $F_1$ is a filter function modeling the delay in entrance of the ambient turbulence into the wake, $F_2$ is a second empirical filter function included to govern the development of the turbulent stresses generated by the wake itself, $k_1$ and $k_2$ are model parameters, and the length scale $l^*$ in this model is taken to be the instantaneous local wake width at which the value is equal to $95\%$ of the ambient freestream velocity.

For this formulation, only the effects from the ambient turbulence intensity and the turbulence generated by the wake itself are included. But the wake development and recovery are also impacted by the ambient atmospheric boundary layer vertical shear. To capture this effect on the wake evolution and recovery, the ambient shear effect is included in the wake deficit model by adding the velocity gradient term due to ambient shear to the original velocity gradient term, which is due to wake deficit mixing \cite{Keck_two}.

\begin{equation}
  \frac{du}{dz}_{ABL}=\frac{u^*_{ABL}}{l^*_{ABL}}
\end{equation}

Where the $\frac{du}{dz}_{ABL}$ is the ambient shear gradient of the atmospheric boundary layer. The ABL velocity scale $u^*_{ABL}$ is based on the neutral atmospheric turbulence spectra and is calculated by the model proposed by Mann \cite{Mann_spatial}\cite{Mann_Wind}. Furthermore, $u^*_{ABL}$ is modified by relating the normal stresses to the shear stresses in the atmosphere through integration of the turbulent energy in the atmospheric turbulence spectra. Different atmospheric conditions would yield different turbulence spectra which then results in various velocity scale values. To estimate the atmospheric length scale, the Monin-Obukhov length is used \cite{Keck_two}.


\section{Initial condition of the wake velocity at the rotor plane} \label{sec:wake_BC}
The wake field is solved using a downstream marching finite difference scheme, in which wake velocity at the crossing plane $P_{n+1}$ is solved based on the information of the previous downstream crossing plane $P_n$. Thus the boundary condition at the rotor plane must be estimated, representing the initial state of the wake development and recovery.

In general, the initial wake velocity at the rotor plane is the induced wind velocity produced by the rotor. However, the near wake in which significant pressure gradients exist is not of primary interest when modelling the effects of wakes and the pressure term in the Navier-Stokes (N-S) equation is omitted in the DWM model. The far wake region beginning at three rotor diameters behind the rotor is referred to as the “point of DWM validity”. As described in section \ref{sec:wake_review}, the wake expands and decelerates in the near wake region, thus omitting the pressure term in the governing equation has the consequence that these effects can not be accurately captured. In the DWM formulation, a filter function with tuned parameters is added to artificially include the wake expansion and the wake velocity deceleration, which is not modeled within the current N-S formulation \cite{hansen_impact_2012}. The filter function is given as:

\begin{equation}
  U_{rot}=U_{amb}[1-(1+f_U)a]
\end{equation}
\begin{equation}
  R_{rot}=R_{rot}\sqrt{\frac{(1-\overline{a})}{[1-(1+f_R)\overline{a}}}
\end{equation}

Where $U_{rot}$ is the waked wind velocity at the rotor plane, $U_{amb}$ is the ambient wind velocity, $R_{rot}$ is the expanded rotor dimension, $\overline{a}$ is the spatial averaged induction factor, $f_U$ and $f_R$ are the tuned parameters based on calibration.


\section{Numerical solution of the wake deficit sub-model governing equations} \label{sec:numerical solution}
The 3-D spatial domain behind the wind turbine consists of multiple 2-D cross-planes whose normal direction is parallel to the rotor axial direction. The 2-D spatial planes are further discretized into a 2-D axisymmetric grid, in which each nodal point is a calculation node where the velocity is calculated by the wake deficit sub-model. The grid of velocities of each cross-plane is resolved from the rotor plane to the downstream cross-planes sequentially.

The momentum equation as shown in Eq. \ref{eq:x_momentum_1} can be rewritten as:
\begin{equation}\label{eq:x_momentum_2}
  U\frac{\partial U}{\partial x}+V\frac{\partial U}{\partial r}=\frac{\nu_t}{r}(\frac{\partial U}{\partial r}+r\frac{\partial^2 U}{\partial r^2})
\end{equation}

\begin{figure}
  \centering
  \includegraphics[scale=0.7]{numerical_scheme}
  \caption{Numerical differencing scheme.}\label{fig:numerical_scheme}
\end{figure}


Here we assume the axial direction from the rotor plane to a downstream location is the positive $i$ direction, and the radial direction from the rotor hub center to the tip is the positive $j$ direction as Figure \ref{fig:numerical_scheme} shows. Thus, using central differencing and upwinding differencing as shown in Eq. \ref{eq:du_dx}, Eq. \ref{eq:du_dr}, and Eq. \ref{eq:du2_dr2} below,
Eq. \ref{eq:x_momentum_2} can be reformulated and yields Eq. \ref{eq:x_momentum_3}.

\begin{equation}\label{eq:du_dx}
  \frac{\partial U}{\partial x}=\frac{U_{i,j}-U_{i-1,j}}{\triangle x}
\end{equation}
\begin{equation}\label{eq:du_dr}
  \frac{\partial U}{\partial r}=\frac{U_{i,j+1}-U_{i,j-1}}{2 \triangle r}
\end{equation}
\begin{equation}\label{eq:du2_dr2}
  \frac{\partial^2 U}{\partial r^2}=\frac{U_{i,j+1}-2U_{i,j}+U_{i,j-1}}{\triangle r^2}
\end{equation}
\begin{equation}\label{eq:x_momentum_3}
  \begin{split}
  (- \frac{V _{i-1,j}}{2\triangle r}+\frac{\nu _t}{2r_j \triangle r}-\frac{\nu _t}{\triangle r^2})U_{i,j-1} +(\frac{U_{i-1,j}}{\triangle x}+\frac{2\nu _t}{\triangle r^2})U_{i,j}+ \\
  (\frac{V_{i-1,j}}{2\triangle r}-\frac{\nu _t}{2r_j\triangle r}-\frac{\nu _t}{\triangle r^2})U_{i,j+1}=\frac{U^2 _{i-1,j}}{\triangle x}
  \end{split}
\end{equation}



Every variable indexed as $i-1$ is known, thus the velocity in the cross-plane $i$ can be efficiently solved by using a tridiagonal matrix solver, such as the Thomas solver \cite{Thomas}. After solving the U velocity component, the radial velocity component V is found from the continuity equation Eq. \ref{eq:continuity_2}, which is derived from Eq. \ref{eq:continuity_1}:
\begin{equation}\label{eq:continuity_2}
V_{i,j+1}=\frac{r_j}{r_{j+1}}V_{i,j}-\frac{\triangle r}{2\triangle x}[(U_{i,j+1}-U_{i-1,j+1})+\frac{r_j}{r_{j+1}}(U_{i,j}-U_{i-1,j})]
\end{equation}

Representative simulation of the wake deficit evolution is shown in Figure \ref{fig:wake_deficit}.

\begin{figure}
  \centering
  \includegraphics[scale=0.44]{single_wake_deficit.png}
  \caption{Wake deficit evolution at various downstream distances}\label{fig:wake_deficit}
\end{figure}

%\subsection{Empirical parameters tuning}


\section{Wake meandering model} \label{sec:wake_meandering}
Wake meandering is the term used to describe the large-scale lateral and vertical movement of the entire wake. Wake meandering is important because it might considerably increase extreme loads and fatigue loads on downstream turbines in wind farms, if the wake is swept in and out of the rotor plane of downstream turbines. Figure \ref{fig:wake_meandering} is a schematic of the wake meandering phenomena, where the yellow color region is the general wake region behind the turbine, and the multiple cross-planes represent the wake trajectory at a single time step.

\begin{figure}
  \centering
  \includegraphics[scale=0.17]{wake_meandering}
  \caption{Schematic of wake meandering.}\label{fig:wake_meandering}
\end{figure}

The phenomenon of wake meandering has long been known empirically but lacked a satisfactory model to capture the movement of the wake. Larsen et al. provides a thorough description of the approach used to model the meandering in the DWM model \cite{Larsen_meandering}. Taylor's frozen turbulence hypothesis is applied for the downstream advection of the wake. Adopting Taylor's hypothesis makes the downstream wake advection only controlled by the local mean wind speed. With this formulation, the wake momentum in the direction of the mean flow is invariant with respect to the prescribed longitudinal wake displacement. The fundamental assumption of this approach is that the wake transport in the atmospheric boundary layer can be modeled by considering the wake to act as a passive tracer driven only by the large-scale turbulence. The properties of the wind field at a certain crossing plane in the meandering frame of reference do not change throughout the whole process, but the exact wake center locations in the fixed frame of reference change with respect to time. The axial speed of this large scale turbulence is assumed to be the ambient mean velocity.

To calculate the wake displacement in the vertical and lateral directions, the wake is modeled as a cascade of wake deficits, each “emitted” at consecutive equally spaced time increments, in agreement with the passive tracer analogy \cite{Larsen_meandering} illustrated in Figure \ref{fig:wake_meandering_inkscape}, where the red curve presents the actual meandered wake center trajectory.
\begin{figure}
  \centering
  \includegraphics[scale=0.49]{wake_meandering_inkscape}
  \caption{Schematic of wake meandering model.}\label{fig:wake_meandering_inkscape}
\end{figure}


At every time step, a cross-plane is released from the rotor plane and marches downstream while the wind properties of this cross-plane are constant during the entire simulation process. The space domain behind the turbine consists of multiple cross-planes, which are orthogonal to the axial wind direction and are all released from the rotor plane with the spacing between two neighboring cross-planes equal to the ambient velocity multiplied by the time step interval. In each cross-plane there is a wake center position. For a single cross-plane, it marches a constant distance downstream at every time step, and a new wake center position is obtained using a low-pass filter function, with the averaged vertical and lateral velocity calculated based on the wake center position of the last time step (i.e. in the closest upstream cross-plane).

As illustrated in Figure \ref{fig:wake_meandering_inkscape}, the space modeled by the wake meandering model behind the turbine is a 3-D domain, with the $x$ direction in the mean flow direction, the $y$ direction in the lateral direction, the $z$ direction in the vertical direction, the investigated turbine located at the cross plane location $A_0$, and the farthest downstream boundary plane at $A_n$. The wake meandering model functions as follows.

First, at time $t_0$, a frozen cross-plane $C_0$ is released from the turbine plane $A_0$ and starts to advect downstream. After time $\triangle t$, this frozen cross-plane $C_0$ arrives at the location of the cross plane $A_1$. This process continues until finally at time $T$ the frozen cross plane $C_0$ reaches the location of the cross-plane $A_n$. Throughout this whole process, at each time step and at each cross plane location $A$, meandered wake center vertical and lateral position coordinates in the frozen cross-plane $C_0$ are returned by applying a low-pass filter, and the wake center coordinate in the x direction is equal to the ambient velocity multiplied by the local total time, which is counted from the original release at the rotor plane. A frozen cross-plane $C$ is released at every time step consecutively and the same approach discussed above is applied for every subsequent frozen cross-plane.

At a cross-plane, the wake center coordinates $(x_{k+1},y_{k+1},z_{k+1})$ of time step $k+1$ are expressed as follows based on the wake center coordinates $(x_{k},y_{k},z_{k})$ of time step $k$:
\begin{equation}
  x_{k+1}=x_k+U\triangle t
\end{equation}
\begin{equation}
  y_{k+1}=y_k+\overline{v_k}\triangle t
\end{equation}
\begin{equation}
  z_{k+1}=z_k+\overline{w_k}\triangle t
\end{equation}
Where $U$ is the ambient wind speed, $\triangle t$ is the time interval, and $\overline{v_k}$ and $\overline{w_k}$ are the spatial averaged vertical and lateral wind velocity over a circular disk whose diameter is 2-D in the frozen cross-plane. That is to say, the movement of the wake is dependent on the atmospheric eddies whose size are equal to or larger than 2-D. Thus the cut-off frequency $f_c$ of the low-pass filter can be specified as:
\begin{equation}
  f_c=\frac{U}{2D}
\end{equation}



\chapter{Creating a wind farm wake model}
When turbines are placed in a row or in a wind farm, the turbines interact with each other from upwind turbines to downwind turbines according to the inflow wind direction. Turbines in the freestream experience the ambient wind speed and turbulence intensity. But for the downstream waked turbines, the inflow is often characterized as a highly turbulent flow with lower mean wind speed compared with the ambient flow. Thus, new inflow conditions need to be defined due to the presence of the wake from the upstream turbines. Section \ref{sec:waked_inflow} discusses the wind velocity for a waked turbine. Section \ref{sec:waked_TI} presents how the turbulence intensity for a waked turbine is modeled. Section \ref{sec:turbine_yaw} discusses the effect of turbine yawing on the wake trajectory. Section \ref{sec:wakes_row} and Section \ref{sec:DWM_windfarm} present how the wakes in a row of turbines and in a wind farm are modeled, respectively.

\section{Inflow wind velocity for waked turbines}\label{sec:waked_inflow}
The inflow wind velocity to the turbine is decomposed into two components, which are the $mean$ $component$ and the $fluctuating$ $component$ as shown below. In general, the mean wind speed or the characteristic wind speed is assumed to be the $mean$ $component$, and the difference between the actual wind speed and the $mean$ $component$ is the $fluctuaing$ $component$.
\begin{equation}
  U=\widetilde{U}+\overline{U}
\end{equation}
For a downwind turbine, the $mean$ $component$ in the inflow wind is not the freestream mean wind speed any more, but it is considered to be the wake deficit velocity which is calculated from the upwind turbine's wake deficit sub-model. There are two reasons that the results of the wake deficit sub-model can be considered to be the $mean$ $component$ of the inflow wind to the downwind turbine. First the wake deficit sub-model solves a steady state N-S equation; second the inlet boundary condition is based on a time averaged turbine induction factor, which will be discussed later.

The $fluctuating$ $component$ for the inflow wind to a waked turbine is modified as follows. On the basis of the fluctuating component of the ambient wind, the contribution of the added turbulence due to the wake from the upwind turbine is included to calculate the added turbulence magnitudes on the downwind turbine.

In general, the modified local wind speed to the downwind turbine $U_{total}(y,z)$ is expressed as below:
\begin{equation}\label{eq:superimpose}
  U_{total}(y,z)=(U_{windfile}(y,z)-U_{amb})*\frac{TI_{added}}{TI_{amb}}+U_{wake}(y,z)
\end{equation}
Where $U_{windfile}(y,z)$ is the wind file velocity at the fixed point $(y,z)$, $TI_{added}$ is the added turbulence intensity due to the wake of upwind turbines, and $U_{wake}(y,z)$ is the wake velocity at the fixed point $(y,z)$ calculated by the wake deficit sub-model.

If there is no wake meandering, and because the output of the wake deficit sub-model is from the steady state N-S equation, then the wake velocity $U_{wake}$ for a certain point $(y,z)$ at a downwind rotor would always be constant with respect to time. However, by introducing the wake meandering sub-model, the wake moves lateral and vertically with respect to time and so the velocity field at a downwind turbine becomes dynamic. The wake velocity calculated by the wake deficit sub-model remains constant, but only in the moving frame of reference. When the quantities in the moving frame of reference are transferred to the fixed frame of reference, the wake velocity $U_{wake}(y,z)$ at a fixed point changes with respect to time due to the wake meandering effect, and so the DWM model produces a dynamic analysis on waked turbines.

\section{Turbulence intensity for waked turbines}\label{sec:waked_TI}
Due to the wake mixing effect and the large scale meandering phenomenon, the turbulence intensity (TI) that the downstream turbines see is no longer the same value that the first turbine experiences. The total turbulence intensity for the downstream turbines includes the wake added TI and the apparent TI, where the wake added TI is due to the velocity gradient of the wake deficit and the apparent TI is due to the wake meandering compared to the fixed rotor fixed frame of reference \cite{Keck_two}.

The turbulence intensity $TI$ is given by:
\begin{equation}
  TI\equiv \frac{u'}{U_{amb}}
\end{equation}
Where $u'$ is the root mean square of the axial velocity fluctuations. And the axial velocity fluctuations are directly correlated with the shear stress, which can be obtained in the wake deficit model by multiplying the velocity gradient and the viscosity. Based on the findings of Keck \cite{Keck_two} and Larsen \cite{Larsen_wake}, the added turbulence intensity generated by the wake itself $TI_{added,wake,1}$ is expressed as:
\begin{equation}
  TI_{added,wake,1}=\sqrt{\frac{1}{C_{u'w'}(w_{rms}'/u_{rms}')}\tau_{stress}}
\end{equation}
Where $C_{u'w'}$ is a correlation factor relating the axial fluctuation and the radial fluctuation, and is dependent on the axial distance in this work. $w_{rms}'$ and $u_{rms}'$ are the root mean square of the axial and radial fluctuations respectively. $\tau_{stress}$ is the shear stress which is calculated by relating the wake velocity gradient and the viscosity as shown in Eq. \ref{eq:shear_stress}.
\begin{equation} \label{eq:shear_stress}
  \tau_{stress}=\nu_t\cdot\frac{du}{dr}
\end{equation}


The combined added turbulence intensity is given below. The reason why a max operator is used is to remove the regions with low shear stress in the wake deficit field, such as the region close to the rotor center line.
\begin{equation} \label{eq:TI_added}
  TI_{added,wake}=max(TI_{added,wake,1},TI_{amb})
\end{equation}

The approach to calculate the apparent added turbulence intensity is more straightforward. By definition, if the velocity variation is divided by the mean velocity, the turbulence intensity is obtained. In this case, the $mean$ $velocity$ is given by the results from the wake deficit sub-model, the velocity variation directly comes from the wake meandering - the wake velocity at a certain location in the fixed frame of reference changes with respect to time due to the wake center movement. The total velocity variation is obtained by integrating the entire local variations.

Due to the split in scales, the added intensity from the wake itself, which has small length scales, are independent on the apparent added intensity from the wake meandering, which has large length scales. The turbulence intensity at the downwind rotor plane is normalized by the local mean velocity instead of the ambient mean wind velocity. The total turbulence intensity on the downwind turbine due to the wake effect of the upwind turbine is found by Eq. \ref{eq:Ti_total}, where $TI_{apparent}$ is the apparent added turbulence intensity, and $U_{LM}$ is the local spatial and time averaged wind velocity at the rotor plane.
\begin{equation}\label{eq:Ti_total}
  TI_{added,total}=\frac{U_{amb}}{U_{LM}}\sqrt{TI_{added,wake}^2+TI_{apparent}^2}
\end{equation}

\section{The effect of turbine yawing on the wake trajectory}\label{sec:turbine_yaw}
A turbine in a wind farm often works in a yawed condition. Yawing a rotor causes a cross-wind component in the thrust force that the rotor exerts on the flow. This results in the deflection of the wake in the opposite direction of the yaw rotation. Many approaches such as vortex theory are applied to quantify the skew angle of the wake behind the rotor plane. In this work, for a yawed turbine the skew angle is modeled based on the findings by Jimenez et al. using momentum conservation theory \cite{Jiménez}. The skew angle $\alpha$ at a certain downstream location $x$ and the lateral displacement $dy$ due to the yaw error $\theta$ are expressed in Eq. \ref{eq:alpha} and Eq. \ref{eq:dy} respectively. $D$ is the rotor diameter, $\delta$ is the local instantaneous wake width and $Ct$ is the rotor thrust coefficient.
\begin{equation}\label{eq:alpha}
  \alpha\approx(\frac{D}{\delta})^2\cos ^2\theta\sin\theta\frac{Ct}{2}
\end{equation}
\begin{equation}\label{eq:dy}
  dy_{yaw}(x)=tan(\alpha)\cdot dx
\end{equation}


\section{Modeling multiple wakes in a row}\label{sec:wakes_row}
For a downstream turbine that experiences waked flow, the aforementioned velocity modification is applied at the load calculation stage carried out by AeroDyn, which is the aerodynamic solver used by the FAST code. The integration of the wind farm model into the NWTC design codes is describer later in chapter \ref{chap:DWM_NWTC}.

The added total turbulence intensity on the downwind turbine is deployed for:
\begin{enumerate}
  \item Scaling the local velocity fluctuation (based on the ambient turbulence intensity) as shown in section \ref{sec:waked_inflow}. This scaled $fluctuating$ $component$ is then superimposed on the downwind turbine inflow wind.
  \item Replacing the ambient turbulence intensity in the wake deficit calculation stage for the downwind turbine. The ambient wind condition for the waked turbine is modified with an increased turbulence intensity. To accurately capture the enhanced wake recovery, the corresponding turbulence intensity is needed as the wake deficit sub-model input.
\end{enumerate}

An important issue to address when modeling wake effects deep inside a wind farm is how to handle wakes from multiple upwind
turbines. This is not straightforward, since the deficit from one turbine interferes with the next turbine and is further
complicated by the turbulent mixing process. In the current work, if a downwind turbine is located in a row in which there are more than one upwind turbines that affect this downwind turbine, the added turbulence intensity and the superimposed waked velocity on the waked turbine are only post-processed from the wake of the $closest$ upwind turbine, or the $strongest$ wake, for the following three reasons:
\begin{enumerate}
  \item While the added TI and the superimposed wake velocity are only post-processed from the wake of the closest wind turbine, this does not mean that contributions from the further upwind turbines are not included. On the contrary, the closest upwind turbine sees the waked flow from these further upwind turbines, which means the operating state of this closest upwind turbine is affected and altered due to the presence of those further upwind turbines, which then impacts the wake from this closest upwind turbine.
  \item From the work of Larsen \cite{larsen_validation_2013}, it is found that the final wake profile under multiple wakes' interference is very similar to the closest upwind turbine's wake for small radial locations. Thus this motivates the approach that only the strongest wake is used when observing wakes from multiple upwind turbines.
  \item The length scales involved in the meandering process are very large, whereas the length scales involved in the wake deficit formulation and recovery are very small. Due to the split of scales, it can be assumed that the wake meandering is independent on the flow disturbances from the other upwind turbines.
\end{enumerate}

\section{Handle multiple wakes in a wind farm}\label{sec:DWM_windfarm}
Turbines are located in a wind farm with multiple rows and arbitrary inflow wind directions. In the software implementation for modeling multiple wakes in a wind farm, turbine first are sorted from upwind to downwind according to the inflow wind direction and the turbine layout. The turbines are simulated consecutively from upwind to downwind in a straightforward manner. The downstream turbine is affected by upwind turbine(s), but not by all of them. Thus it is crucial to determine if a downstream turbine is affected by particular upwind turbines, based on the instantaneous inflow wind direction. Then the corresponding added turbulence intensity and the decreased wake speed may be superimposed on the downwind turbine.

A pre-screening process to identity which downwind turbine is affected by which upwind turbines is required. An upwind turbine only affects a downwind turbine if the downwind turbine is located in a sector area behind the upwind turbine. To help quantify the sector area, the $sector$ $angle$ is introduced. The sector angle is expressed as the angle between the two edges of the wake at a certain downstream location, and includes the contribution from both the wake expansion and wake meandering. The sector angle is a function of the ambient turbulence intensity and the downstream distance as is shown in Figure \ref{fig:sector angle}. The lateral distance between the two edges of the wake at a certain downstream location is obtained by looping through all of the simulation time and is assigned as the maximum value of the lateral displacement of the wake edges over all time. Repeating this procedure through each downwind location yields the wake sector angle as a function of the downstream distance. This approach to calculate the sector angle is conservative since only the maximum value of the wake center displacement is employed.

\begin{figure}
  \centering
  \includegraphics[scale=0.45]{wake_sector_angle}
  \caption{Wake sector angle as a function of TI and downstream distance.}\label{fig:sector angle}
\end{figure}

Based on the wake expansion and turbine layouts, the main criteria to specify which upstream turbines affect which downstream turbines are listed as follows:
\begin{itemize}
  \item A downwind turbine will only be influenced by the wake from upwind turbines.
  \item Two angles are compared: the angle between the line connecting the two turbines and the wind direction line, and the wake sector angle at this downstream rotor location. If the former is larger than the latter, then the downwind turbine will see the freestream wind velocity, and vice versa.
\end{itemize}


\chapter{WFMP: Incorporating the wind farm wake model into the NWTC design codes}\label{chap:DWM_NWTC}
The aforementioned wake model is a stochastic wind field simulator, and is able to model the wakes of wind turbines. The NWTC design codes suite, which includes FAST, TurbSim, and AeroDyn, is a comprehensive time-domain fully-coupled aero-hydro-servo-elastic simulator capable of simulating horizontal-axis wind turbines (HAWTs). In this work, to combine the features of both tools, the wind farm model is incorporated within the NWTC design codes suits, thus enabling the investigation of the power production and the loads variation of the turbines in the waked flow. The combination of the two tools makes a comprehensive program called the Wind Farm Modeling Program (WFMP).

Within WFMP, a driver program is created to pre-screen the wind farm topology and post-process the outputs generated from each turbine. WFMP is incorporated into both FAST V7 and FAST V8, with the FAST V8 version released by NREL and could be downloaded at https://nwtc.nrel.gov/DWM.

\section{Wind Farm Modeling Program and FAST}
In WFMP, there are seven major steps to implement the wind farm model into FAST as listed below and are presented in Figure \ref{fig:DWM_work_flowchart}.
\begin{figure}
  \centering
  \includegraphics[scale=0.65]{DWM_work_flowchart}
  \caption{Wind Farm Modeling Program main steps.}\label{fig:DWM_work_flowchart}
\end{figure}


\begin{enumerate}
  \item To create the stochastic random wind field impacting the wind turbine, TurbSim generates a full 4-D wind field with user defined characteristic wind speed and turbulence intensity. The choice of the turbulence spectrum to create this wind field is not exclusive, but in this work the Kaimal spectrum is chosen.
  \item Obtain the initial induced wake velocity at the rotor plane, by running a full FAST simulation with AeroDyn using the BEM method to return the time-averaged axial induction factors at each spanwise node of the blade. If the turbine does not see the freestream velocity, but instead is a downwind turbine that sees the waked flow from upwind turbines, a velocity superimposition procedure is performed in the AeroDyn subroutine, in which the upstream meandered wake deficit profile is superimposed on the freestream wind as shown in Eq. \ref{eq:superimpose}.
  \item Calculate the wake deficit field using the approach discussed in section \ref{sec:wake_eqn}.
  \item Pass the calculated wake width at each downstream location from the wake deficit model to the wake meandering model. The wake width is used as the size of the filter, which returns the spatial averaged vertical and lateral large scale wake movement speed.
  \item Calculate the wake meandering using the methods discussed in the section \ref{sec:wake_meandering}.
  \item If this investigated turbine's wake affects a downwind turbine, the enhanced turbulence intensity at the downstream rotor plane with respect to the meandered wake center is calculated using Eq. \ref{eq:TI_added} to determine the single wake boundary condition for the downwind turbine.
  \item For the next turbine, the steps $(2)$-$(6)$ are repeated.
\end{enumerate}

In the second step, the induction factors are calculated at each annulus of the blade, one can directly obtain the induction values from the AeroDyn output. However, what is really needed is a measurement of the momentum transfer between the rotor and the flow. The axial induction factors given by AeroDyn must be interpreted carefully, as the tip loss correction in AeroDyn causes the calculated induction values near the tip to be too large and unphysical. The induction factor can be treated as the proxy for the momentum transfer between the rotor and the flow, but only when there are no tip losses. The thrust coefficient is a direct measurement of the momentum transfer and thus is a physically appropriate measurement to use. The steps to calculate the induction factor are listed below:
\begin{enumerate}
  \item Obtain the thrust force of each section of the blade directly from the AeroDyn output. From basic momentum theory, the thrust force already reflects the blade tip loss effect.
  \item Calculate $C_t$ (the thrust coefficient) from Eq. \ref{eq:Ct}. $F$ is the thrust force, $\rho$ is the air density, $A$ is the area of annulus, $u$ is the local wind velocity and $i$ symbolizes the $i^{th}$ annulus.
      \begin{equation}\label{eq:Ct}
      Ct_i=\frac{F_i}{\frac{1}{2}\rho A_i u_i^2}
      \end{equation}
  \item Use the calculated $C_t$ values to determine the value of the induction factor by solving Eq. \ref{eq:induction}, which is derived from simple momentum theory.
      \begin{equation}\label{eq:induction}
      Ct_i=4a_i(1-a_i)
      \end{equation}
  \item Use these modified values of induction factors as the inputs to the wake deficit model using the techniques discussed in section \ref{sec:wake_BC}.
\end{enumerate}

\section{Wind Farm Modeling Program and driver program}\label{sec:DWM_FAST}
In the Wind Farm Modeling Program, the module of the single wake model is configured as a sub-module of AeroDyn. When running WFMP, the wind farm model is run at the same time as the running of FAST, and vice versa. To simulate the wake behind a single turbine, a single instance of the turbine is running. If a row of wind turbines or a wind farm needs to be simulated, multiple consecutive independent turbines are manipulated to run sequentially, which is accomplished by introducing a driver program discussed later. A diagram of the input/output for a single turbine is shown in Figure \ref{fig:DWM_IO}. For each turbine, the inputs of the wind farm model come from two sources - the AeroDyn loads results and the wind farm model input text files, while the outputs consists of the wake deficit velocity and the meandered wake center locations in the field behind the investigated turbines. If the turbine receives a waked flow, a third input that includes the decreased local wind velocity and enhanced turbulence intensity from the upstream wake is added; if the turbine's wake affect a downstream turbine, the model outputs will include the enhanced turbulence intensity at the corresponding downstream turbine location.
\begin{figure}
  \centering
  \includegraphics[scale=0.8]{modified_dwm_IO}
  \caption{Single turbine inputs and outputs.}\label{fig:DWM_IO}
\end{figure}

Due to the fact that the turbines are running sequentially and separately, they use written files to pass the data from one turbine to another. A driver program is created to manipulate the simulation of multiple consecutive independent turbines in a wind farm as shown in Figure \ref{fig:driver program flowchart}. The order of the turbines simulated is from the upwind turbines to the downwind turbines sequentially, and is sorted based on the inflow wind direction and the wind turbine layout. To read the model inputs and the wind farm specifications, a text file similar to the FAST input files is built and read by the driver program. Table \ref{t:Driver_input} shows the input parameters as well as corresponding explanations of the input text file of the driver program, which includes the wind farm turbine layout, inflow wind direction and the wind farm model parameters. Detailed explanations are included in Appendix A. The driver program is written in Fortran and can easily be transferred to be written in other computer languages.
\begin{figure}
  \centering
  \includegraphics[scale=0.8]{DWM_driver_program_flowchart}
  \caption{Driver program flowchart.}\label{fig:driver program flowchart}
\end{figure}





\begin{table}[ht]
\caption{Driver program inputs}
\label{t:Driver_input}
\centering
\begin{tabular}{c c}
\hline\hline
Variable Name & Variable Explanation\\ [0.5ex]
\hline
HubHt & hub height (m)\\
RoterR & rotor radius (m)\\
NumWT & total number of wind turbines (-)\\
Uambient & ambient wind velocity (m/s)\\
TI & ambient turbulence intensity (\%)\\
ppR & number of points per radius (-)\\
Domain\_R & radial domain size (R)\\
Domain\_X & longitudinal domian size (R)\\
Meandering\_simulation\_time\_length &total time steps for a fixed cross plane  \\
Meandering\_Moving\_time &total time steps for a moving cross plane \\
WFLowerBd & The lower bound of the wind file (m)\\
Winddir & ambient wind direction (degree)\\
RanW & Random walk model flag (-)\\
DWM.exe name & file rootname of the DWM-FAST program (R)\\
XCoordinate & x-coordinate of a turbine (R)\\
YCoordinate & y-coordinate of a turbine (R)\\ [1ex]
\hline %inserts single line
\end{tabular}
\label{table:nonlin} % is used to refer this table in the text
\end{table}


To conclude, the driver program has three objectives:
 \begin{enumerate}
   \item The turbines are simulated separately and sequentially, so the driver program is used to manipulate the simulation of multiple consecutive independent turbines in a wind farm. The total number of instances of simulation that are run is equal to the total number of investigated turbines.
   \item The appropriate order of the turbine analysis is from the upwind turbines to the downwind turbines. The driver program reads in the wind farm turbine coordinate layout and the inflow wind direction, calculates the projected distance of the wind turbine coordinates on the inflow wind direction, and sorts/labels the turbines from upwind to downwind. The upwind wind turbines are simulated first and the downwind turbines are simulated under the influence of the upwind turbines.
   \item In a wind farm, a downwind turbine is affected by some upwind turbines, but not by all of them. Thus, after sorting the wind turbines, the driver program applies the approach discussed in section \ref{sec:DWM_windfarm} to determine which downwind turbines are affected by which upwind turbine(s).
 \end{enumerate}


\chapter{Simulation results of the Wind Farm Modeling Program}
In this chapter, the performance of WFMP is validated and calibrated in terms of waked turbine power and loads, and the wake advection speed is investigated. Section \ref{sec:row_results} discusses the turbine performance in a straight row when applying WFMP as the simulation tool. Section \ref{sec:wind_farm_results} presents the comparison of the turbine power production in a wind farm. In section \ref{sec:DWM_sensitivity}, the wind farm model's sensitivity to the input wind file is investigated. By investigating the wake files and using Genetic Algorithm, the wake advection speed is modeled in section \ref{sec:DWM_advection_time}.

\section{Simulation and validation of the Wind Farm Modeling Program for a row of wind turbines}\label{sec:row_results}
In this section, WFMP is used to simulate the wakes of turbines in a row for the North Hoyle wind farm and Lillgrund wind farm. The wind farm model results are validated against high fidelity OpenFOAM LES and the field data in terms of turbine power and blade loads \cite{Churchfield_LES}. Moreover, comparisons for a row of yawed turbines are also included.

\subsection{North Hoyle wind farm case}
There are $3$ separate rows of turbines in the North Hoyle wind farm. The turbine spacings in each row are 10D, 11D, and 4.4D, and the total number of turbines in each row are 4, 4, and 5 respectively. In the OpenFOAM LES the wind direction used for validation is straight down the row (no angle offset), the mean wind velocity is 8m/s and the turbulence intensity is 6\% at the hub-height \cite{Churchfield_LES}. To minimize the uncertainty when comparing the results achieved by two different models, five wind files are generated in TurbSim and used in the DWM wind farm model simulations. These wind files have the identical mean wind speed and TI as the LES case, but different random seed numbers that ensures identical wind statistics but random varying turbulence. The final wind farm model results are obtained by averaging the results across the different wind file cases.

A generic Vestas $V80$ $2MW$ turbine is modeled in FAST and employed as the test turbine in the North Hoyle wind farm \cite{Churchfield_turbine}. The turbine uses a simple region 2 torque control law, in which the generator torque is proportional to the generator speed squared, as shown in Eq. \ref{eq:torqeu_control}. The value of the constant $K_{V80}$ is set to be $0.0016551$ $N\cdot m/RPM^2$.
\begin{equation}\label{eq:torqeu_control}
  T_{gen}=K_{V80}\Omega _{gen}^2
\end{equation}

Figure \ref{fig:power_north_hoyle_1} below shows the normalized power production for the turbines of 11D, 10D and 4.4D spacing respectively. The power production of downstream turbines is normalized by the first turbine. The time-averaged power production of the wind farm model is obtained directly from the AeroDyn output files. The results show a good agreement between the wind farm model and the OpenFOAM LES. A large power drop is seen from the $1^{st}$ turbine to the $2^{nd}$ turbine, which is due to the wake influence by the upstream 1st turbine. Moving down the row, the TI behind the $2^{nd}$ turbine becomes larger than that behind the 1st turbine due to the enhanced shear layer mixing, which results in a faster wake recovery behind the $2^{nd}$ turbine and so the $3^{rd}$ turbine experiences a larger wind velocity than the $2^{nd}$ turbine. An equilibrium power production is reached from the $3^{rd}$ turbine onwards. The extent of the wake recovery is proportional to the turbine spacing, if all other parameters are kept the same. This explains why in Figure \ref{fig:power_north_hoyle_1} the normalized power production for the cases with larger spacing is greater than those with smaller spacing. The difference of the normalized power between the wind farm model with different input wind files and the OpenFOAM LES is shown in Figure \ref{fig:power_north_hoyle_2}, from which it can be seen that the variability in the results from using different wind files is negligible.
\begin{figure}
  \centering
  \includegraphics[scale=0.38]{power_north_hoyle_1}
  \caption{North Hoyle wind farm normalized power production with different turbine spacings.}\label{fig:power_north_hoyle_1}
\end{figure}

\begin{figure}
  \centering
  \includegraphics[scale=0.33]{power_north_hoyle_2}
  \caption{North Hoyle wind farm normalized power difference in detail.}\label{fig:power_north_hoyle_2}
\end{figure}

Unlike other models that describe the wake effect either on the power production or on the loading aspects, the wind farm model provides an opportunity to consider both impacts by incorporating the wind farm model into FAST. The loads on a single blade are compared between the wind farm model and OpenFOAM LES in terms of the mean and standard deviation of the axial force per unit length. In Figure \ref{fig:loads_north_hoyle_1} the mean axial force per unit length for the 11D spacing case is shown, from which it can be seen that the wind farm model matches the high fidelity OpenFOAM LES results well, which is expected due to the fact that a good agreement is shown in the previous power production comparison. However, as shown in Figure \ref{fig:loads_north_hoyle_2}, the wind farm model model underestimates the loads standard deviation for the turbines downstream of the first turbine compared to OpenFOAM LES.
\begin{figure}
  \centering
  \includegraphics[scale=0.45]{loads_north_hoyle_1}
  \caption{North Hoyle wind farm mean blade loads of 11D spacing.}\label{fig:loads_north_hoyle_1}
\end{figure}

\begin{figure}
  \centering
  \includegraphics[scale=0.42]{loads_north_hoyle_2}
  \caption{North Hoyle wind farm blade loads standard deviation of 11D spacing.}\label{fig:loads_north_hoyle_2}
\end{figure}

The loads standard deviation of the downstream turbines is highly affected by the meandered wake center from the upwind turbine. To locate the wake center in the OpenFOAM LES, the instantaneous wake profile at each downstream location is correlated with a Gaussian profile, and the location of maximum correlation gives the centerline. Thus the standard deviation of the meandered wake center is investigated and compared between the wind farm model and the OpenFOAM LES for the three spacing cases in Figure \ref{fig:meandering_north_hoyle}. It can be seen that the variability of the wake center is underestimated in the wind farm model.
\begin{figure}
  \centering
  \includegraphics[scale=0.45]{meandering_north_hoyle}
  \caption{Wake center standard deviation comparison.}\label{fig:meandering_north_hoyle}
\end{figure}

Two conclusions can be drawn from these results. First, it appears that the TurbSim generated wind files have insufficiently large length scales to accurately capture the wake meandering, as it is substantially underpredicated compared to OpenFOAM LES. An alternative approach is to directly generate wind files by sampling from the OpenFOAM LES data. This approach has been tested, and appears to result in wind files with sufficiently large length scales such that the meandering variability is accurately calculated in the wind farm model. An alternative approach is to modify the wind farm model, such that the wake center meandering amplitude at each time step is artificially increased by inserting a correction factor that causes the wind farm model to have the same wake center standard deviation as the OpenFOAM LES.

Meandering statistics are important for loads analysis, especially for the fatigue analysis and loads variation investigations. The main advantage of the wind farm model compared with other similar tools is that the large scale wake movement contribution is included, which enables dynamic analysis on the downwind turbines. The wake meandering effect and statistics will be further validated and calibrated in the future.

\begin{figure}
  \centering
  \includegraphics[scale=0.65]{added_ti_meadering}
  \caption{Loads standard deviation with added TI superimposition onto the downwind turbine.}\label{fig:added_ti_meadering}
\end{figure}

Second, the wake added turbulence plays an important role in calculating the loads standard deviation on turbines. Eq. \ref{eq:superimpose} may be used to superimpose the added turbulence onto the downwind turbine. The loads variation is shown in Figure \ref{fig:added_ti_meadering}, from which it can be seen that the loads standard deviation is increased due to the enhanced turbulence superimposition.

The modified loads standard deviation comparison is presented in Figure \ref{fig:loads_meandering_modificaion}, demonstrating that by doing those modification, the loads variation yielded by the wind farm model matches well with the OpenFOAM LES.
\begin{figure}
  \centering
  \includegraphics[scale=0.57]{loads_meandering_modificaion}
  \caption{Modified loads standard deviation comparison between the wind farm model and OpenFOAM LES.}\label{fig:loads_meandering_modificaion}
\end{figure}



\subsection{Lillgrund Wind Farm Case}
The layout of the Lillgrund wind farm is shown in Figure \ref{fig:Lillgrund}. The analysis presented here is carried out for row B (turbines 15-8) and row D (turbines 30-24). Row B contains eight turbines that have an identical spacing of 4.4D; however the turbine that would have been the $4^{th}$ turbine in row D is omitted from the actual wind farm because the water is too shallow to allow access by construction boats, thus the spacing between the 3rd and the $4^{th}$ turbines is 8.8D, and all the other spacings are 4.4D in row D. The ambient wind flows from the left bottom corner to the right top corner perfectly along the rows.
\begin{figure}
  \centering
  \includegraphics[scale=0.6]{Lillgrund}
  \caption{The layout of the Lillgrund wind farm.}\label{fig:Lillgrund}
\end{figure}

The two rows in the Lillgrund wind farm are simulated with a freestream wind with a mean hub height velocity of 9m/s and turbulence intensity of 6.2\%. Like the North Hoyle wind farm case, $5$ TurbSim generated wind files that have the identical wind speed and TI but different random seed numbers are applied in the Lillgrund wind farm case. The Siemens SWT-2.3-93 $2.3$ MW turbine, again applying a simple region 2 torque control law where the value of the constant $K_{SWT}$ is set to be $0.0045619$ $N\cdot m/RPM^2$, is modeled in FAST and employed as the test turbine for the Lillgrund wind farm case.

The normalized power comparison results over the two rows are shown in Figure \ref{fig:power_Lillgrund}, where the top plot is of row B and the bottom plot is of row D. The wind farm model results are obtained directly from the AeroDyn output and compared to the OpenFOAM LES results simulated by Churchfield et al., and the field data is presented by Dahlberg\cite{Churchfield_LES}. At the $4^{th}$ turbine in row D, where there is a larger separation from the upstream turbine, the DWM model correctly captures the enhanced wake recovery that results in the larger power production for the $4^{th}$ turbine. The main deviation seen from the field data is the power production for the $6^{th}$, $7^{th}$ and $8^{th}$ turbine, where the OpenFOAM LES and the wind farm model both overestimate the rotor power production. The reasons for this phenomenon are not clear now, but may be due to the boundary layer stability, deep array effect or yaw error of the turbines. The power difference between the wind farm model and OpenFOAM LES as well as the field data are shown in Figure \ref{fig:power_Lillgrund_DWM_LES} and Figure \ref{fig:power_Lillgrund_DWM_field}, respectively.
\begin{figure}
  \centering
  \includegraphics[scale=0.4]{power_Lillgrund}
  \caption{Lillgrund wind farm row B (top) and row D (bottom) power production.}\label{fig:power_Lillgrund}
\end{figure}
\begin{figure}
  \centering
  \includegraphics[scale=0.4]{power_Lillgrund_DWM_LES}
  \caption{Lillgrund wind farm normalized power difference between DWM and OpenFOAM LES for row B (top) and row D (bottom).}\label{fig:power_Lillgrund_DWM_LES}
\end{figure}
\begin{figure}
  \centering
  \includegraphics[scale=0.4]{power_Lillgrund_DWM_field}
  \caption{ Lillgrund wind farm normalized power difference between DWM and field data for row B (top) and row D (bottom).}\label{fig:power_Lillgrund_DWM_field}
\end{figure}


\subsection{Yawed Turbine Case}
A two-turbine case is performed by the high fidelity SOWFA OpenFOAM simulation \cite{Fleming_2013}. The identical setup applied in the SOWFA simulation is used in the wind farm model to simulate the turbine performance in a yawed condition. The turbine deployed in this case is the NREL 5MW wind turbine and a variable-speed control is applied\cite{5MW}. The mean wind velocity is 8m/s and the TI is 6\%. The yaw angles for the front turbine are $-25^\circ$, $0^\circ$ and $25^\circ$ respectively, and the $2^{nd}$ turbine rotor axial direction is always in line with the inflow wind direction($0^\circ$ yaw).

It is found by Fleming et al. that the wake moves to the right with increasing downstream distance due to the rotor rotational effect, even when there is no yaw misalignment\cite{Fleming_redirecting}. Thus, in the SOWFA simulation for the $-25^\circ$ and $25^\circ$ yaw case, the power results at the downstream turbine differ noticeably due to the discrepancy in the wake trajectory offset behind the front turbine. However in the current wind farm model, this wake-shift effect has not been included, so the results of the $-25^\circ$ and $25^\circ$ are taken as the averaged value for both the wind farm model and SOWFA simulation as shown in Table \ref{tab:yaw_power} in terms of the normalized power of the downwind turbine. It can be seen that the normalized power at the downstream turbine for the zero yaw case of the wind farm model is almost equal to the SOWFA result, and for a yawed front turbine the wind farm model results in general agree well with the SOWFA OpenFOAM simulation.
\begin{table}[h] \label{tab:yaw_power}
\caption{Normalized power of the downstream turbine with a yawed front turbine}
\centering
\begin{tabular}{c rr c}
\hline\hline
&\multicolumn{2}{c}{Normalized downstream turbine power} & \\ [0.5ex]
\hline
Front turbine yaw angle&SOWFA&DWM&Difference\\
\(\displaystyle 0 ^\circ \)&0.4343&0.4358&-0.0015\\
\(\displaystyle 25 ^\circ \)&0.7946&0.6533&/ \\
\(\displaystyle -25 ^\circ \)&0.6397&0.6566&/ \\
\(\displaystyle Average\ 25 ^\circ \)&0.7171&0.6550&0.0622\\
\hline % inserts single-line
\end{tabular}
\label{tab:hresult}
\end{table}




\section{Simulation and validation of the Wind Farm Modeling Program for a wind farm}\label{sec:wind_farm_results}
The wind farm model is compared with and validated against LES and field data from the Egmond aan Zee (OWEZ) offshore wind plant, for which comprehensive field data is available. Substantial analysis has been performed on the full set of field data from the wind plant, including strain-gauge measurements of mechanical loads on two of the turbines. The OWEZ wind plant, which is roughly 10 km off the shore of The Netherlands, consists of 36 Vestas
V90-3.0MW wind turbines with 70 m towers as shown in Figure \ref{fig:owez}. The turbines are situated in four
major rows. The turbine spacing within a row is roughly 7 rotor diameters (D) and the spacing between rows is roughly 11D. Turbines 7 and 8 are fully instrumented for mechanical loads measurements. A meteorological mast with various wind and water sensors is situated to the southwest of turbines 7 and 8. The meteorological mast is useful for quantifying the wind plant inflow when the winds are from the south- southeast through the west-northwest.
\begin{figure}
  \centering
  \includegraphics[scale=0.9]{owez}
  \caption{Layout of the OWEZ wind farm}\label{fig:owez}
\end{figure}

The Wind Farm Modeling Program described in section \ref{sec:DWM_windfarm} is applied to this wind farm simulation case. The driver program pre-screens the wind farm turbine layout, the inflow wind direction, and manipulates the sequentially running of the wind farm model for each wind turbine from upwind to downwind. 

We are interested in wind-plant performance over the entire wind rose, and turbulence-intensity range. The results of WFMP are compared with the results of LES and field data. Many more cases are able to be simulated with the WFMP than with LES because of WFMP’s relative inexpensiveness. The data are split into different turbulence intensity ranges and all ranges are simulated with the WFMP. For the 4-6\% turbulence intensity range, the entire wind direction range are simulate with WFMP, but for all other turbulence intensity ranges, the wind direction range about the main row direction, which is $318.7^\circ$, is focused on. In addition, because of its high cost, only one case with LES is simulated. A summary of the cases simulated is shown in Figure \ref{fig:wf_all_case}. All the inflow wind to WFMP are generated by TurbSim.

\begin{figure}
  \centering
  \includegraphics[scale=0.4]{wf_all_case}
  \caption{A summary of the cases simulated}\label{fig:wf_all_case}
\end{figure}

The field data consists of 10-minute statistics over three years. Each WFMP simulation was run for 350 s, the statistics of which were found to be converged, alleviating the need to run WFMP for the full 10 minutes. The LES was run for over 10 minutes, but statistics are taken from the last 10 minutes of the simulation.

Both the power and loads are focused on, so the results are split into two subsections. For power, the global wind plant efficiency are examined. For loads, the blade-root out-of-plane (BR-OOP) bending moment is solely focused on.


\subsection{Wind Farm Power and Efficiency}
In comparing wind plant power and efficiency field data to simulation data, these quantities versus wind direction are often plotted. The field data is comprised of three years of 10-minute averages, and there is significant scatter in the data. To better understand the field data dependence on wind direction, it is binned in $5^\circ$ bins. Additionally, the field data direction measurement contains uncertainty, as discussed by Gaumond et al \cite{Gaumond}. When examining power and efficiency in a turbine-row-aligned direction, the wind direction uncertainty of the field data makes the wake effects appear weaker. For example, if the mean wind direction of a single 10-minute data point is shown as row aligned, but the actual direction is offset from the row direction by a few degrees, or the wind swept across the row direction over the ten minutes, the efficiency will appear higher than if the wind were actually row aligned.

It quickly became apparent that one cannot simply compare the simulation output to the field data without accounting for the fact that the field data contains wind direction uncertainty. Therefore, the method of Gaumond et al.\cite{Gaumond} is used. In which the time-averaged simulation data as a function of wind direction is convolved with a Gaussian function of wind direction centered on the wind direction of interest. This means that for each wind direction, multiple simulations are required at and around the wind direction of interest. The convolution is done for each wind direction. The convolution replaces data at a specific wind direction with a Gaussian-weighted average of that data point and the surrounding data points. Gaumond et al. explain that the Gaussian width, $\sigma_a$, is directly related to the uncertainty of the wind direction within the 10-minute period over which the average was taken. That wind direction uncertainty can be measured by the standard deviation of the wind direction over the 10-minute period.

Because the data is examined as a function of wind direction, but binned by turbulence intensity, and turbulence intensity affects the standard deviation of wind direction, the 10-minute derived standard deviation of wind direction versus turbulence intensity from meteorological mast data taken at hub height over the three years of available data is plotted in Figure \ref{fig:TI_windSTD} \cite{dwm_les_owez}. Then a linear fit on this data is performed to define a function of wind direction standard deviation versus turbulence intensity. The wind direction standard deviation as a function of turbulence intensity is used as the Gaussian-weighting function width $\sigma_a$. The linear fit is given by Eq. \ref{eq:GaussWidth}, where TI is the turbulence intensity. 

\begin{equation}\label{eq:GaussWidth}
  \sigma_a=0.88+39.23TI
\end{equation}

\begin{figure}
  \centering
  \includegraphics[scale=0.7]{TI_windSTD}
  \caption{10-minute derived standard deviation of wind direction versus turbulence intensity}\label{fig:TI_windSTD}
\end{figure}

For the field data, the wind-direction-bin average is performed, so the same must be done to the simulation data in order to make a useful comparison. The effect of bin-averaging is to cause wake effects to appear further reduced. The effect of Gaussian-weighting and wind-direction-bin averaging is shown in Figure \ref{fig:DWM_gauss}.

\begin{figure}
  \centering
  \includegraphics[scale=0.6]{DWM_gauss}
  \caption{The effect of applying the Gaussian weighting and wind-direction-bin averaging to the DWM model data.}\label{fig:DWM_gauss}
\end{figure}

Figure \ref{fig:OWEZ_DWM_power} presents the normalized power production for each turbine with a $315^\circ$ inflow wind direction by using WFMP where wake effects on the downstream turbines are clearly shown.
\begin{figure}
  \centering
  \includegraphics[scale=0.6]{OWEZ_DWM_power}
  \caption{Normalized power production computed by the wind farm model.}\label{fig:OWEZ_DWM_power}
\end{figure}

The wind farm efficiency with different turbulence intensities is examined in Figure \ref{fig:TI_24}, \ref{fig:TI_46}, \ref{fig:TI_68}, \ref{fig:TI_810}, and \ref{fig:TI_1012}. Each figure corresponds to a different turbulence intensity range. Wind plant efficiency is the total power generated by the wind plant divided by the power achieved if all 36 turbines were in the freestream and not subject to wake effects. Freestream power is defined as the mean of the 10-minute time-averaged power generated by the leading turbines. The blue symbols represent different data points from the field data that fit within the 8 - 10 m/s wind speed bin and the different turbulence intensity bins. The black line is the wind-direction-bin average of these field data points within $5^\circ$ bins, but portions of the line are omitted if fewer than 3 data samples occupy a given wind-direction bin. The red and green symbols represent the DWM model and LES data points, respectively, with no direction-uncertainty Gaussian weighting. The cyan line is the $5^\circ$ wind- direction-bin average of the Gaussian-weighted DWM model data. The vertical black dotted lines indicated wind directions aligned with a turbine row, and the turbine spacing for each of these row directions is given in terms of rotor diameters.

\begin{figure}
  \centering
  \includegraphics[scale=0.6]{TI_24}
  \caption{Wind plant efficiency versus wind direction with 2\%-4\% TI}\label{fig:TI_24}
\end{figure}

\begin{figure}
  \centering
  \includegraphics[scale=0.6]{TI_46}
  \caption{Wind plant efficiency versus wind direction with 4\%-6\% TI}\label{fig:TI_46}
\end{figure}

\begin{figure}
  \centering
  \includegraphics[scale=0.6]{TI_68}
  \caption{Wind plant efficiency versus wind direction with 6\%-8\% TI}\label{fig:TI_68}
\end{figure}

\begin{figure}
  \centering
  \includegraphics[scale=0.6]{TI_810}
  \caption{Wind plant efficiency versus wind direction with 8\%-10\% TI}\label{fig:TI_810}
\end{figure}

\begin{figure}
  \centering
  \includegraphics[scale=0.6]{TI_1012}
  \caption{Wind plant efficiency versus wind direction with 10\%-12\% TI}\label{fig:TI_1012}
\end{figure}

Decreases in efficiency observed in the field data are clearly seen along row directions, especially the 7.1 D direction of $318.7^\circ$. Very little field data is available when wind is aligned with this row from the opposite direction. For the lowest turbulence intensity bin, 2\% - 4\%, individual 10-minute average data points show that the efficiency drops to as low as 0.5 for the $318.7^\circ$ direction (but the $5^\circ$-bin average is roughly 0.65). As the turbulence-intensity bin level is increased, this minimum efficiency increases. For example, for the 8\% - 10\% turbulence intensity bin, the field data bin average efficiency for the $318.7^\circ$ direction is near 0.8. For directions in which the spacing is larger, the drop in efficiency is much smaller. For example, in the 2\% - 4\% turbulence intensity bin, the 11.1 D spacing efficiency is roughly 0.85, compared to 0.65 for the 7.1 D spacing.

The WFMP reflects the field data reasonably well. Decreases in efficiency are observed when wind is aligned with rows, and the efficiency is lower for small spacings. The individual WFMP data points, the weighted, and the bin-averaged data lie well within the scatter of the field data.

In regions of the wind rose in which wake effects are not present, the field data has considerable scatter above and below one, but the wind-direction-bin average is below 1. The values that are less than 1 could be present because there may be wind direction and speed variations across the wind plant that allow for waking in some parts of the plant. For instance, when the meteorological mast-measured direction is one in which wake effects should be absent, the turbines near the meteorological mast likely experience that non-waking wind direction, but with increasing distance from the mast, wind direction uncertainty increases and there is a higher probability that the wind direction does cause some degree of waking. The greater than 1 values may be present because of wind speed variations across the farm such that the majority of the wind plant is experiencing higher wind speeds than the freestream turbines. For WFMP, does not model such stochastic variations across the wind plant. All turbines are subject to the inflow wind files generated by TurbSim with same mean wind speed and turbulence intensity.

As only one LES case was run due to its expense, there is one LES data point on the 4\% - 6\% turbulence intensity plot at a wind direction of $315^\circ$. Qualitatively, it lies well within the field data scatter and the results of WFMP.

It is also useful to look at the power produced by individual turbines relative to the freestream turbines. Figure \ref{fig:row_1}, \ref{fig:row_2}, \ref{fig:row_3}, and \ref{fig:row_4} show the relative power of each turbine in each of the four rows when the wind is from $315.7^/circ$ and the turbulence intensity is 4\% - 6\%. The wind direction is $4^/circ$ from row-aligned. For the field data, the mean of all 10-minute averages that lie within the row direction $\pm1^\circ$ is shown, and the gray shaded area shows $\pm1$ standard deviation of this collection of 10-minute average for each turbine. Because the WFMP was not run for three years like the field data, but rather each distinct condition was simulated once, the standard deviation of the WFMP data are not shown. However, the WMFP data is weighted with the Gaussian function, as discussed above, and wind-direction-bin averaged using a $\pm1^\circ$ bin as with the field data. The LES results are included as well.

\begin{figure}
  \centering
  \includegraphics[scale=0.4]{row_1}
  \caption{Mean power of each turbine in row 1 relative to the average of the power produced by the freestream turbine}\label{fig:row_1}
\end{figure}

\begin{figure}
  \centering
  \includegraphics[scale=0.4]{row_2}
  \caption{Mean power of each turbine in row 2 relative to the average of the power produced by the freestream turbine}\label{fig:row_2}
\end{figure}

\begin{figure}
  \centering
  \includegraphics[scale=0.4]{row_3}
  \caption{Mean power of each turbine in row 3 relative to the average of the power produced by the freestream turbine}\label{fig:row_3}
\end{figure}

\begin{figure}
  \centering
  \includegraphics[scale=0.4]{row_4}
  \caption{Mean power of each turbine in row 4 relative to the average of the power produced by the freestream turbine}\label{fig:row_4}
\end{figure}

Both WFMP and LES are shown to underpredict the relative power compared with the field data. However, they both lie well within one standard deviation of the field data. The underestimation is not a clear indicator that both models are performing poorly. It is believed that the wind direction uncertainty in the field data, causes the wake losses to appear artificially weak.  

This results present the challenges of designing an apples-to-apples comparison between the field data and simulation tools. Some of the downwind turbine power of row 1 and row 4 are even larger than 1 - which means their power is larger than the freestream turbine. This is entirely possible because wind speed varies across the farm such that those turbines are experiencing higher wind speeds than the freestream turbines. The power underestimation by both of the simulation tools might be due to wind speed uncertainty in the wind farm.

It is also found that WFMP results are highly sensitive to wind-direction-bin averaging width. Using a bin width of $\pm2^\circ$ significantly decreases the apparent wake losses, bringing the model results more in line with the field data. In addition, the WFMP simulations shown here were at a single freestream turbulence intensity of 4.5\%, but the field data is comprised of measurements when turbulence intensity is anywhere between 4\% and 6\%. The 4.5\% is below the midpoint of the field data turbulence intensity range, and for lower turbulence intensity, greater power deficits are expected. 

The WFMP does predict important features of the power losses down the row. For example, it does predict the drop in power in the second turbine in a row followed by some degree of recovery. This behavior is commonly seen in field data, and is clearly visible in the field data for rows 1 and 4. It is not visible in the field data from rows 2 and 3, and one possible reason may be that the turbines are not located exactly as stated. If there is relatively minor lateral offset of the second turbine in the row, the dramatic power decrease for the second turbine may not occur. WFMP also well predicts the power recovery at turbines 16, 24, and 31 due to the increased spacing between those turbines and the next turbines upstream.

The effect of wind turbine spacing on relative power can also be examined because the OWEZ wind plant has turbine spacings of 7.1 D, 11.1 D, 13.2 D, and 18.1 D. Figure \ref{fig:D71}, \ref{fig:D111}, \ref{fig:D132}, and \ref{fig:D181} present relative mean power for different spacings. The field data is from the 4\% - 6\% freestream turbulence intensity bin, and the WFMP data is from a simulation with 4.5\% turbulence intensity. The field data is wind-direction-bin averaged to $\pm1^\circ$ of the row directions, and the WFMP data is Gaussian weighted and also wind-direction-bin averaged to $\pm1^\circ$. WFMP generally slightly underpredicts the power deficit of the downstream turbines. However, the prediction lies will within the field data scatter. Also, WFMP results well follow the trend of the field data in that the power deficit of the downstream turbines is decreased as spacing is increased.

\begin{figure}
  \centering
  \includegraphics[scale=0.4]{D71}
  \caption{Turbine mean power of turbine spacing 7.1D}\label{fig:D71}
\end{figure}

\begin{figure}
  \centering
  \includegraphics[scale=0.4]{D111}
  \caption{Turbine mean power of turbine spacing 11.1D}\label{fig:D111}
\end{figure}

\begin{figure}
  \centering
  \includegraphics[scale=0.4]{D132}
  \caption{Turbine mean power of turbine spacing 13.2D}\label{fig:D132}
\end{figure}

\begin{figure}
  \centering
  \includegraphics[scale=0.4]{D181}
  \caption{Turbine mean power of turbine spacing 18.1D}\label{fig:D181}
\end{figure}


\subsection{Mechanical Loads}
In this section the blade-root out-of-plane (BR-OOP) bending moment is examined. This quantity is a popular choice in examining turbine structural response to its inflow because it is linked with blade fatigue, fatigue of the connection of the blade to the hub, and it causes a non-torque moment on the main shaft if the blades are not all experiencing the same BR-OOP bending moment at the same time.

Figure \ref{fig:loads_TI_24}, \ref{fig:loads_TI_46}, \ref{fig:loads_TI_68}, \ref{fig:loads_TI_810}, and \ref{fig:loads_TI_1012} show the normalized mean BR-OOP bending moment on the blades of turbine 7 over the entire wind rose for different turbulence intensity levels and in the 8 m/s - 10 m/s wind speed bin. The regions of waking are clearly indicated by decreased bending moment because the rotor experiences a lower wind speed. These plots closely follow the behavior of wind plant efficiency shown in the previous section. For wind directions in which the turbine spacing is greater, the minimum mean BR-OOP bending moment higher than for closer spacings, as is expected due to increased wake recovery at greater distances. Also, elevated turbulence intensity makes the minima have higher values. The results of WMFP matches well with the LES results and field data.

\begin{figure}
  \centering
  \includegraphics[scale=0.6]{loads_TI_24}
  \caption{Turbine 7 normalized BR-OOP bending moment versus wind direction with 2\%-4\% TI}\label{fig:loads_TI_24}
\end{figure}

\begin{figure}
  \centering
  \includegraphics[scale=0.6]{loads_TI_46}
  \caption{Turbine 7 normalized BR-OOP bending moment versus wind direction with 4\%-6\% TI}\label{fig:loads_TI_46}
\end{figure}

\begin{figure}
  \centering
  \includegraphics[scale=0.6]{loads_TI_68}
  \caption{Turbine 7 normalized BR-OOP bending moment versus wind direction with 6\%-8\% TI}\label{fig:loads_TI_68}
\end{figure}

\begin{figure}
  \centering
  \includegraphics[scale=0.6]{loads_TI_810}
  \caption{Turbine 7 normalized BR-OOP bending moment versus wind direction with 8\%-10\% TI}\label{fig:loads_TI_810}
\end{figure}

\begin{figure}
  \centering
  \includegraphics[scale=0.6]{loads_TI_1012}
  \caption{Turbine 7 normalized BR-OOP bending moment versus wind direction with 10\%-12\% TI}\label{fig:loads_TI_1012}
\end{figure}

\subsection{Calculation Cost}
The WFMP and LES have dramatically different computational costs. The WFMP simulations were run on a desktop computer with two 2.0 GHz cores. 100 different 350-s WFMP simulations of the entire OWEZ wind plant required roughly 24 hours of compute time on this desktop. In contrast, a single 350-s LES of the entire OWEZ wind plant requires 280,000 CPU-hrs on a high-performance computing system.

\section{The sensitivity of Wind Farm Modeling Program to input wind files}\label{sec:DWM_sensitivity}
Wind Farm Modeling Program uses a wind file as an input, either generated by TurbSim or resolved by OpenFOAM. Previous results in section \ref{sec:row_results} demonstrated that the discrepancy in the DWM model results is negligible when reading in various wind files that have identical mean wind speed and turbulence intensity. However, the boundary condition of the wake deficit model is the time-averaged induced velocity at the rotor plane, averaged over the total simulation time length. It is unknown that what is the appropriate or adequate time averaging window that ensures that the results are not be affected by the window length. This issue is more important when dynamic changes in the turbine operating state are considered.

The correlation between the \emph{averaging window length}, which is used to filter the time averaged induced velocity at the rotor plane, and the wake deficit profile at downstream locations is investigated. Specifically, three scenarios of the “time window” are investigated:
\begin{enumerate}
  \item Different wind files are used.
  \item Different time window lengths are applied, e.g., 20s, 30s, and 100s.
  \item Different time window locations are used, e.g., 20-50s, 100-130s, and 200-230s.
\end{enumerate}

\subsection{Different wind files}
The comparison of the wake profiles obtained by using three different wind files at several different downstream locations is shown in Figure \ref{fig:wake_comparison_wind_file_1}. These TurbSim generated wind files have identical mean wind speed and turbulence intensity, but different random seed numbers. The time-averaged absolute difference over each radial nodal point from hub center to the 1D radial location compared between the three wind files is shown in Figure \ref{fig:wake_comparison_wind_file_2}. The difference between the results shows that the effect of different wind files on the wake deficit profile is negligible.
\begin{figure}
  \centering
  \includegraphics[scale=0.45]{wake_comparison_wind_file_1}
  \caption{Wake profiles of 3 different wind files.}\label{fig:wake_comparison_wind_file_1}
\end{figure}
\begin{figure}
  \centering
  \includegraphics[scale=0.45]{wake_comparison_wind_file_2}
  \caption{Radial nodal mean absolute difference between different wind files.}\label{fig:wake_comparison_wind_file_2}
\end{figure}


\subsection{Different time window lengths}
The comparison of the wake profiles obtained by using three different time window lengths at several different downstream locations is shown in Figure \ref{fig:wake_comparison_length_1}. Besides the time window lengths, all the other inputs are identical. Furthermore, the size of the time window is reduced to 20 seconds, and the time-averaged absolute difference over each nodal point from hub center to the 1D radial location between the 20-90s time window and 600s time window is plotted in Figure \ref{fig:wake_comparison_length_2}.
\begin{figure}
  \centering
  \includegraphics[scale=0.45]{wake_comparison_length_1}
  \caption{Wake profiles of 3 different time window sizes}\label{fig:wake_comparison_length_1}
\end{figure}
\begin{figure}
  \centering
  \includegraphics[scale=0.33]{wake_comparison_length_2}
  \caption{Radial nodal mean absolute difference between different time window sizes}\label{fig:wake_comparison_length_2}
\end{figure}


\subsection{Different time window locations}
The comparison of the wake profiles obtained by using three different time window locations at several different downstream locations is shown in Figure \ref{fig:wake_comparison_window_location_1}. Furthermore, the time-averaged absolute difference over each nodal point from hub center to the 1D radial location between three different time window locations is plotted in Figure \ref{fig:wake_comparison_window_location_2}.
\begin{figure}
  \centering
  \includegraphics[scale=0.45]{wake_comparison_window_location_1}
  \caption{Wake profiles of 3 different time window locations}\label{fig:wake_comparison_window_location_1}
\end{figure}
\begin{figure}
  \centering
  \includegraphics[scale=0.45]{wake_comparison_window_location_2}
  \caption{Radial nodal mean absolute difference between different time window locations}\label{fig:wake_comparison_window_location_2}
\end{figure}

\subsection{Conclusions}
For all the three scenarios, the mean wake deficit nodal differences between two different settings are very small - around 1\%-3\%, even for the extreme 20 second time window case. There is no monotonic correlation between the wake deficit profile and the time-window size. The wake deficit results further can be assumed to be independent on the start time, i.e., during a simulation the time-window can be equally chosen anywhere but the rotor transient stage.

Furthermore, it is found that these conclusions are still valid no matter the values of the wind speed and turbulence intensity. Based on these findings, the influence of the variations of the time-window sizes, locations and different wind files on the wake deficit results can be assumed to be negligible, which resolves our concern when applying the wind farm model to a real-time system.

\section{Wake advection time (speed) analysis}\label{sec:DWM_advection_time}
Our future goal is to deploy the wind farm wake model to help control turbines dynamically, thus a prerequisite problem that needs to be answered is how fast the wake advects, or how long it will take for a wake from the upwind turbine to reach a downwind turbine so that the downwind turbine’s control scheme may be adjusted accordingly.

The wake advection speed at a certain downstream location is expressed as the spatial averaged velocity over a disk circle whose radius is \emph{r}. Thus a model is built using Genetic Programing (GP) to estimate the filter size \emph{r} at any downstream location based on the comparison and calibration between the DWM and the high fidelity OpenFOAM simulation.

The filter size is found to be dependent on the turbulence intensity, local downstream position, upwind turbine’s thrust coefficient, and the ambient wind speed. By using this model, the wake advection time between two turbines can be accordingly predicted and estimated. The filter size equation obtained via GP is shown in Eq. \ref{eq:GP}.

\begin{equation}\label{eq:GP}
\begin{split}
r=46.94368*exp((0.11624*TI)/(TI-16.66576*exp(x)))*(0.317734 \\
/(exp(0.5868*x)*(Average_{Ct} - Wind_U/TI))- TI/(2.01794*Wind_U \\
+ 2.01794*exp(0.0802634*Wind_U)) + (6.348199*Average_{Ct})/(TI \\
*Wind_U) +(1.61443*Average_{Ct}*TI)/(x + 2.32652) - 0.838013) \\
 - 23.6991*exp((TI*x)/log(Average_{Ct})) + 73.92
\end{split}
\end{equation}



The comparison between the original data and the GP model estimation, and the comparison between the original data and the $3^{rd}$ order linear model estimation are shown in Figure \ref{fig:3rd_order} and Figure \ref{fig:GP} respectively. The coefficients of determination of the GP model and the $3^{rd}$ order linear model are \emph{96.82\%} and \emph{92.70\%} respectively. From the figures it also can be seen that the GP model yields closer results when the original date series are at the peaks or the bottoms where there are higher change rate.

\begin{figure}
  \centering
  \includegraphics[scale=0.5]{3rd_order}
  \caption{Comparison between $3^{rd}$ order linear model and original data.}\label{fig:3rd_order}
\end{figure}

\begin{figure}
  \centering
  \includegraphics[scale=0.5]{GP}
  \caption{Comparison between GP model and original data.}\label{fig:GP}
\end{figure}


\chapter{Improvements to the Wake meandering model}
Wake meandering is the term used to describe the large-scale lateral and vertical movement of the entire wake. Wake meandering is important because it might considerably increase extreme loads and fatigue loads on downstream turbines of a wind farm, if the wake is swept in and out of the rotor plane of downstream turbines. In this chapter, Section \ref{sec:general_meandering_model} addresses the issue of how to accurately model the wake meandering. It first reviews the limitations of the current wake meandering method, and then introduces another model called the Random Walk (RW) model. Section \ref{sec:rw_model}   further discusses the details of the RW model, and its tuning, training, and validation. Section \ref{sec:RW_detail} and \ref{sec:timeComplexity} discuss the implementations of both models in the DWM code and compare the performance in terms of the meandering statistics and the computational cost. Section \ref{sec:meandering_recom} compares and summarizes the two models, and provides recommendation for utilizing the program.

\section{Modeling of wake meandering}\label{sec:general_meandering_model}
\subsection{Low-pass filter model}
A wake is characterized by a mean wind velocity decrease and turbulence increase behind a turbine. Also the wake center moves both laterally and vertically when it marches downstream. To include this effect in the DWM program, a method based on Taylor's hypothesis was applied and is discussed thoroughly in section \ref{sec:wake_meandering}. 

The fundamental assumption of this approach is that the wake transport in the atmospheric boundary layer can be modeled by considering the wake to act as a passive tracer driven only by the large-scale turbulence. The properties of the wind field at a certain crossing plane in the meandering frame of reference do not change throughout the whole process while this crossing plane is marching downstream. Based on this, a cascade model could be used to track the wake center at each downstream location.

This model is regarded as a ``low-pass'' filter model and it is based on a pre-generated wind file (in this case, a TurbSim generated wind file). For each downstream location and each time step, it calculates an instantaneous spatially averaged velocity as the large scale wake advection speed in the lateral and vertical directions respectively, hence obtaining the wake center location at the next time step by incorporating the advection time. This approach is independent of any LES of field data results for tuning the model.

To evaluate the performance of this model, the wake center time series is compared with the results of LES simulation using the North Hoyle wind farm case. To locate the wake center in the OpenFOAM LES, the instantaneous wake profile at each downstream location is correlated with a Gaussian profile, and the location of maximum correlation gives the wake centerline. The comparison of the wake center position standard deviation at downstream turbine planes for three different spacing cases is shown in Figure \ref{fig:meandering_north_hoyle1}. The variability of the wake center movement is underestimated by the ``low-pass'' filter model. 

It appears that the TurbSim generated wind files have insufficiently large length scales to accurately capture the wake meandering, as it is substantially underpredicated compared to OpenFOAM LES. An alternative approach is to directly generate wind files by sampling from the OpenFOAM LES data. This approach has been tested, and appears to result in wind files with sufficiently large length scales such that the meandering variability is accurately calculated in DWM. However, generating the wind files by sampling LES data for each wind condition is not practical from a computational perspective, especially when a wide range of wind speeds and turbulence intensities are being analyzed.  

\begin{figure}
  \centering
  \includegraphics[scale=0.45]{meandering_north_hoyle}
  \caption{Wake center standard deviation comparison.}\label{fig:meandering_north_hoyle1}
\end{figure}

The underestimation of the meandered wake center variation is directly reflected in the variation of the thrust force on the downstream turbine blade as shown in Figure \ref{fig:2_loads_north_hoyle_2}. It can be seen that the loads variation of the downstream turbines modeled using the ``low-pass'' filter model is underestimated compared with the OpenFOAM LES simulation results. To address this issue, an alternative approach is to utilize a correction factor that artificially increases the wake center meandering amplitude at each time step in the ``low-pass'' model, such that the variation of the meandered wake center would be comparable with the LES results. When this approach is implemented and the correction factor is tuned, the loads variation is enhanced and is close to the results of LES as shown in Figure \ref{fig:1_loads_meandering_modificaion}.

\begin{figure}
  \centering
  \includegraphics[scale=0.42]{loads_north_hoyle_2}
  \caption{North Hoyle wind farm blade loads standard deviation of 11D spacing.}\label{fig:2_loads_north_hoyle_2}
\end{figure}

\begin{figure}
  \centering
  \includegraphics[scale=0.57]{loads_meandering_modificaion}
  \caption{Modified loads standard deviation comparison between the wind farm model and OpenFOAM LES.}\label{fig:1_loads_meandering_modificaion}
\end{figure}

The low-pass filter method may be regarded as a ``physics based'' approach, because it does not depend upon the LES or field results in advance and is solely based on the input wind file.

\subsection{Random Walk model}
Because the physics-based model performance is validated statistically eventually, it might be possible to directly model the meandered wake center sequence solely based on the statistics without using any wind profiles. A statistical method called ``random walk'' model is created and applied to directly model the meandered wake center positions. There are two advantages of this kind of model. First, it is cost-efficient, because looping over the wind file to calculate the spatial average velocity is expensive both in time and memory. Second, the proposed model is flexible because it is easy to modify and can be applied to any case as long as a probability distribution is known. 

A random walk is a mathematical formalization of a path that consists of a succession of random steps. For example, the path traced by a molecule as it travels in a liquid or a gas, the search path of a foraging animal, the price of a fluctuation stock can all be modeled as random walks, although they may not be truly random in reality. The term random walk was first introduced by Karl Pearson in 1905 \cite{Pearson}.

Often random walks are assumed to be Markov chains. A Markov chain named after Andrey Markov, is a random process that undergoes transitions from one state to another on a state space. It must possess a property that is usually characterized as ``memorylessness'' \cite{JamesR}: the probability distribution of the next state depends only on the current state and not on the sequence of events that preceded it. This specific kind of ``memorylessness'' is called the Markov property. A Markov chain is a sequence of random variables $X_1$, $X_2$, $X_3$, ... with the Markov property, namely that, given the present state, the future and past states are independent. 

\begin{equation}\label{eq:markov}
 Pr=(X_{(n+1)} = x \mid X_{(n)}=x_{(n)})
\end{equation}

When thinking about the process of wake meandering, a ``cascade model'' could be utilized - at a discrete time and location, the current wake center is moved from the wake center location of last discrete time and upstream location. Thus when using the terms of random walk model, the current state would be referred to the current wake center, and the next state would be the wake center of the next step.

The wake meandering is a random process, but within a range. For example, in the lateral direction, the wake center could reach a very far away location from the centerline, but with only a very small probability. Likewise, the wake center could reach a location that is close to the centerline with a much higher probability. When using random walk model to describe this, the transition probability is the probability that the wake center reaches the current(or next) location having the wake center of previous (or current) state, similar to Eq. \ref{eq:markov} defines. 
  
The fundamental reason why random walk model could be applied to model the wake meandering is because the process of wake meandering and the random walk model are both random and their evaluation criteria are similar. 

For the wake meandering, it is not possible to predict the exact location of a wake center that will happen every single time, instead, no matter what method is used to generate the meandered wake center series, eventually they will be compared with LES data or field data statistically by using mean or standard deviation. 

Likewise, when the problems are resolved using the random walk, since the system changes randomly, it is generally impossible to predict with certainty the state of a Markov chain at a given point in the future. However, the statistical properties of the system's future can be predicted. In this case, the overall mean wake center location and wake center variation are the properties that being predicted and compared.

Compared with the traditional low-pass filter method, random walk model has two advantages. First, it is cost-efficient, it does not need to loop over the wind file to calculate the spatial average velocity which is expensive in both time and memory. Second, this model is easy to modify since it is a parameter-based approach.  
  
\section{Random Walk Model}\label{sec:rw_model}
In section \ref{sec:general_meandering_model}, the random walk model is introduced in general. In this section, the technical detail of random walk model will be discussed first, followed by the tuning, training, and validation of the random walk model.

\subsection{Random walk model}
Random walk model uses a Markov chain process to model the wake meandering between two neighboring states. The main assumption of Markov chain process is that the probability distribution of the next state depends only on the current state and not on the sequence of events that preceded it. When applying Markov chain to the wake meandering, the wake center movement at current step is only determined by the current wake center location.

In many high fidelity LES simulation cases, to determine the meandered wake center, the instantaneous wake profile at each downstream location is correlated with a Gaussian profile, and the location of maximum correlation gives the wake centerline. Thus the meandered wake center series is assumed to conform to a Gaussian probability distribution with the mean $\mu$ and the standard deviation $\sigma$. This probability distribution is used to train the random walk model and simulate the meandered wake center. The probability distribution is not limited to Gaussian, and could be any kind of probability distribution as long as the mean and standard deviation are known.

The random walk models how the wake center moves between two neighboring steps (states). Suppose the wake center moves on a 1D straight line (it represents the wake center movement in vertical or lateral direction). The main logic behind determining which direction the wake center will move is straightforward and is inspired by the fact that the wake center is less likely to move to a very far away location compared with a location close to the turbine centerline.

As mentioned earlier by the Markov chain theory, how the wake center moves solely depends on the current wake center location. To determine which direction the wake center is more likely to move to, with the aid of the probability distribution, the distance of the current wake center between the centerline in the probability distribution (here a Gaussian distribution is used) is quantified. 

The current wake center location $p_n$ is mapped onto the x-axis of the probability distribution. Then the Cumulative Distribution Function (CDF) of $p_n$ is calculated as $\Theta_n$. If $\Theta_n$ is larger than $1-\Theta_n$, then the wake center is more likely to move towards the direction which the CDF covers. For example, if the current location is more towards the distribution edges, then the wake center is more likely to move towards centerline; and if the current location is at the centerline, then there would be equal possibilities that the wake center moves to left or right. Eventually, the wake center bounces back and force randomly in the possibility distribution space and more importantly in the real space which correlates to the possibility distribution. Thus the meandered wake center locations are simulated.

During the whole process of random walk, for the wake center movement, only ``more likely'' or ``less likely'' is mentioned. Incorporated with a randomized algorithm, the proposed random walk model employs a degree of randomness as part of the logic - by introducing a random number and compare it to $\Theta_n$ and thus determine which direction the wake center will eventually move to. Due to the fact that this is a randomized algorithm but not a deterministic algorithm, if the model is trained well, with sufficient amount of steps, this model would yield the expected performance.

The workflow of the random walk model are listed below step by step.

\begin{enumerate}
  \item Initially, the wake center vertical direction domain or lateral direction domain is mapped to the 1D probability distribution space with the mean $\mu$ and the standard deviation $\sigma$. The location $L_0$ of the wake center in real space is set to be at the hub centerline, and in the probability distribution space, its corresponding mapped point $p_0$ is set to be at the probability distribution center (which is the origin $x=0$ point). From here, $p_0$ either moves to the left or right with a step size $d_0$. If this probability distribution is from the vertical(lateral) wake center movements, then the wake center $L_0$ would moves either up(left) or down(right) with a step size $d_0$.
  \item Calculate the area $a_1$ under the curve to left of $p_0$ as $\Theta_0$ (CDF). Then the area to the right of $p_0$ becomes to be $1-\Theta_0$.
  \item $p_0$ will either move to the left or right. The probability that $p_0$ moves to left or right corresponds to $\Theta_0$ and $1-\Theta_0$ respectively.
  \item Generate a random number $R$ between 0 and 1. If $R$ is smaller than $\Theta_0$, then the wake center will move to the left with a step size $d_0$. Otherwise, it will move to the right with a step size $d_0$. Also update the corresponding wake center in the real space.
  \item Then the wake center reaches a new location $L_1$, with $p_1$ the corresponding x coordinate in the probability distribution.
  \item Repeat steps 2-5 for the following time steps until the end.      
\end{enumerate}

If the time steps are sufficient, the meandered wake centers will have the mean $\mu$ and the standard deviation $\sigma$ as expected. For the step size $d$, it could be assumed to be constant if the time step size is small enough. 

\subsection{Model training, tuning, and validation}
Random walk model is a statistical model and is parameter based. The ultimate goal of this model is to produce a series of wake center locations that match the LES results in terms of mean and standard deviation. Moreover, just having a same mean and standard deviation does not guarantee that the random walk model results would have the same behavior compared with the LES results. Also, just replicating the histogram would not guarantee similar wake center sequence is produced. Because wake meandering is a continuous process, different meandered wake center sequences which shares a similar histogram would possibly have very different impacts on the downwind turbines in terms of power and loads.

For example, a step-response function would have the same mean and standard deviation compared with a sine wave, but when applying those signals onto a turbine, the turbine would have totally different responses in terms of the load and fatigue. Therefore, the results from random walk model must also match the LES results in terms of autocorrelation.

It is observed that directly using the random walk data would result in a very high autocorrelation value, because the difference of two neighboring steps are very slight thus strongly correlated. To resolve this issue, a technique called ``thinning'' is applied \cite{thinning}.

Here is how thinning works. Consider a chain that is strongly autocorrelated, one way to decrease autocorrelation is to thin the sample, using only every $n^{th}$ step. To get the $s$ kept steps in a thinned chain, $n*s$ steps in total needs to be generated. The thinned autocorrelation value depends on the thinning factor $n$, more severe the autocorrelation is, the longer the chain needs to be. In this case, a empirical value of 40 is applied as the thinning factor.         

Then the two parameters of the random walk model are the wake center step size between two neighboring steps, noted as \textit{m}, and the scaled standard deviation in the probability distribution space, noted as \textit{n}. Thus the model training and tuning task becomes to be an optimization problem:

\begin{enumerate}
  \item Ensures that the standard deviation of the wake center series from the random walk model matches the one from LES as close as possible. 
  \item Ensures that the autocorrelation of the wake center series from the random walk model matches the one from LES as close as possible. 
\end{enumerate}

To train and tune the random walk model, one thing to note is that the random walk model is not a deterministic model but rather a randomized algorithm model, thus the quantities quantifying this model are in terms of the ``mean'' or expected value, which is calculated based on the results from sufficient steps. Thus the metrics to quantify the standard deviation and the autocorrelation of the wake center series are chosen to be the Mean Squared Error (MSE) between the multiple results of random walk model and the LES resutls. The mean squared error of the standard deviation is noted as $Q_1$. The mean squared error of the summation of the first five second’s autocorrelation value is noted as $Q_2$.

Since the random walk model is a 2-parameter-based model, to find the optimum parameters, the grid search will be applied. Listed below are the steps of performing the grid search. 

\begin{enumerate}
  \item First generate a vector of \textit{m} and \textit{n} respectively as the candidates, the grid search will be performed in the 2D space defined by the \textit{m} and \textit{n} vector.
  \item For each parameter set (\textit{m}, \textit{n}), run the random walk model with this parameter setting for 100 times.
  \item After step 2 is finished, for each set of parameters, obtain $Q_1$ and $Q_2$.
  \item Repeat steps 2 and 3 for each parameter set (\textit{m}, \textit{n}) and obtain its corresponding $Q_1$ and $Q_2$.
  \item The minimum value $Q_1$ and $Q_2$ corresponds to the optimum parameter set of (\textit{m}, \textit{n}).
\end{enumerate}

The grid search results in terms of the normalized MSE of the standard deviation ($Q_1$) is shown in Figure \ref{fig:gridSeachSD}. The x-axis corresponds to 8 different walk step size values in the vector list, and the y-axis corresponds to 7 different probability distribution standard deviation values. In this case, a Gaussian distribution is applied. From the results it can be seen that the minimum value is at the upper left corner.

\begin{figure}
  \centering
  \includegraphics[scale=1.0]{grid_search_MSE_standard_deviation}
  \caption{Grid search in terms of the normalized MSE of the standard deviation.}\label{fig:gridSeachSD}
\end{figure}

The grid search results in terms of the normalized MSE of the autocorrelation values ($Q_2$) is shown in Figure \ref{fig:gridSeachAC}. It can be seen that the minimum value is at the lower right corner. 

\begin{figure}
  \centering
  \includegraphics[scale=1.0]{grid_search_MSE_autocorr}
  \caption{Grid search in terms of the normalized MSE of the autocorrelation.}\label{fig:gridSeachAC}
\end{figure}

From the results shown in Figure \ref{fig:gridSeachSD} and Figure \ref{fig:gridSeachAC}, at the diagonal direction from upper left to lower right, the value of the MSE is monotonically increasing and monotonically reducing respectively. Thus it is impossible to locate a parameter set (\textit{m}, \textit{n}) that produces the minimum error for both autocorrelation and standard deviation.

To locate the optimum parameter set (\textit{m}, \textit{n}), $Q_1$ and $Q_2$ are combined, and the objective function is defined in the Eq. \ref{eq:argmin}, where $Q_1$ and $Q_2$ are both normalized values.

\begin{equation}\label{eq:argmin}
  (m,n) = \argmin_{i,j\in[1,N]} (Q_1(m_i,n_j) + Q_2(m_i,n_j))
\end{equation}

The grid search in terms of the total objective function is shown in Figure \ref{fig:objective}. The minimum value is located at the lower right corner, where corresponds to the optimum parameter set of (\textit{m}, \textit{n}).

\begin{figure}
  \centering
  \includegraphics[scale=1.0]{grid_search_total_objective_function}
  \caption{Grid search in terms of the total objective function.}\label{fig:objective}
\end{figure}

One thing to note here that the training and tuning of the random walk model is based on the LES results, thus for different LES data sets, the approaches of training and tuning the model would be the same, but the optimum parameter value would not be guaranteed to be the same. In this case, the LES results is sampled from the NREL 5MW wind turbine simulation with a wind speed 8m/s and an approximately TI 10\%.

To visualize the model training and tuning results, the histogram and the autocorrelation of the wake centers of the random walk model and the LES are compared. In Figure \ref{fig:les_hist} and \ref{fig:rw_hist}, the wake center histogram from the LES results and the random walk model are shown respectively. In Figure \ref{fig:autocorr_comparison} the autocorrelation of the random walk model and the LES are compared. From both of the comparison, a great agreement between the random walk model results and the LES is shown.

\begin{figure}
  \centering
  \includegraphics[scale=0.9]{LES_histogram}
  \caption{LES wake center histogram.}\label{fig:les_hist}
\end{figure}

\begin{figure}
  \centering
  \includegraphics[scale=0.9]{random_walk_histogram}
  \caption{Random walk model wake center histogram.}\label{fig:rw_hist}
\end{figure}

\begin{figure}
  \centering
  \includegraphics[scale=0.9]{autocorr_comparison}
  \caption{Autocorrelation comparison.}\label{fig:autocorr_comparison}
\end{figure}

\subsection{Model validation}
After training the random walk model, it is necessary to validate this model to ensure that it produces expected performance. A good agreement has already be seen in the autocorrelation and histogram comparison.

Since this is not a deterministic algorithm but rather a randomized algorithm, it is expected that with sufficient steps, the meandered wake centers would have the mean and standard deviation as expected. Thus in order to validate this model, the number of steps that are sufficient to produce the expected result are defined empirically.

To obtain the relationship between the total steps and the model performance, a total step size vector from 100 to 20000 with an increment 100 steps is created. For each value in the vector, the average standard deviation of the model results are calculated from 100 repeated runs, and compared with the expected standard deviation.

This validation result is shown in Figure \ref{fig:empirical}. The blue horizontal line is the base value, which is also the expected value. The red curve is the average standard deviation of the random walk model of each different total walk steps. From the results it can be seen that the random walk model results match the baseline when the total walks steps reach 7,000. Considering there is a 40 thinning factor, thus the minimum required steps would be around 175. 

\begin{figure}
  \centering
  \includegraphics[scale=0.8]{empirical_choose_optimal_mean}
  \caption{Walk steps vs. standard deviation.}\label{fig:empirical}
\end{figure}

\subsection{Grid Search for the wake of the downstream turbine}
The grid search and model training mentioned in the previous sections are based on the wake of the $1^{st}$ turbine from LES results. In this section, the grid search and model training of the random walk model will be carried for the wake from the downstream turbine of the LES results. The downstream turbine is under the wake of the upwind turbine, the wind direction is perpendicular to the rotor plane, and the turbine spacing is 7D. 

The model parameters are still the same and the objective function to train the model would be the same as well. However, instead of a full matrix search, based on the findings from the previous training experience, the parameter search is performed on a tridiagonal matrix from the upper left to the lower right.  
 
The model training is based on the LES results. After post-processing the meandered wake centers of both the upwind and downwind turbines from the LES results, there are mainly two findings and are as listed below.

\begin{enumerate}
  \item The wake center time series standard deviation of the upwind turbine and the downstream turbine for the selected downstream locations are very close.
  \item The autocorrelation of the wake center location time series from the downstream turbine is higher than that of the free stream turbine. This might be due to the fact that the wake meandering of the downwind turbine is much slower compared with the upwind turbine, thus the difference between two neighboring steps are slightly smaller than that of the upwind turbine, therefore causing a higher autocorrelation value.
\end{enumerate}

Based on the LES results, the random walk model is trained for the wake from both the upwind turbine and the downwind turbine. The MSE of the autocorrelation error of the main diagonal are shown in Figure \ref{fig:diag_corr}. Where the red curve represents the wake from the upwind turbine, and the blue curve represents the wake from the downwind turbine. The y axis is the MSE of the autocorrelation error, and the x axis represents different walk step sizes in a descending order ( i.e. the larger the index, the smaller the walk step size).

\begin{figure}
  \centering
  \includegraphics[scale=0.60]{diagonal_autocorr_two_turbines}
  \caption{MSE of the autocorrelation error for the upwind and downwind turbines.}\label{fig:diag_corr}
\end{figure}

From the results it can be seen that, in the left half of the plot, to reach the same amount of MSE, the walk step size of the downwind turbine is smaller than that of the upwind turbine. Since the smaller the step between two neighboring steps, the stronger they correlate. This finding from the random walk model matches the LES finding that the true autocorrelation value of the wake from the downwind turbine is larger than than of the upwind turbine. From the figure, it also can been seen that the red curve bounces back at the right half. This is due to the fact that MSE quantifies the ``absolute distance'' between the true value and the predicted value and does not care about the ``direction''.  

The training of the random walk model combines the quantification results of both the wake center standard deviation and the autocorrelation. Since for the upwind turbine and downwind turbine, there are not too much difference in the standard deviation, the training results between the two turbines mainly depend on the autocorrelation difference between the wakes of the two turbines. A schematic plot of the training results comparison between the upwind turbine and downwind turbine is shown in Figure \ref{fig:rw_two_turbine} in which the optimum parameter sets for both cases are shown respectively. It can be seen that the model for the downstream turbine has a smaller step size thus producing a higher autocorrelation value. We did know that the random walk model for the downwind turbine should yield a smaller step size value, and since the grid search is in a tridiagonal matrix, the location of the other parameter could be quickly located. 

\begin{figure}
  \centering
  \includegraphics[scale=0.65]{RW_two_turbine}
  \caption{The random walk model grid search for both the upwind and downwind turbines.}\label{fig:rw_two_turbine}
\end{figure}

\section{Model the meandered wake centers in DWM}\label{sec:RW_detail}
Both of the two models - a physical low-pass filter model and a statistical random walk model have been implemented into the DWM-FAST program to model the meandered wake center. Previously, the DWM results of using the low-pass filter model has been validated. In this section, the DWM results using random walk model will be compared against the one using the low-pass filter model.

In this test case, two 5MW NREL turbines are deployed with a spacing 7D. The wind speed is 8m/s and the TI is 10\%. The wind turbines aer simulated with FAST-DWM with 300 seconds. The low-pass filter model and the random walk model are used separately to simulate the meandered wake center thus the results of the two models are compared. In the low-pass filter, a scale factor is applied.

First, the meandered wake center locations in the lateral direction of 100 seconds at the same downstream distance are compared between the two models. The results are shown in Figure \ref{fig:wake_center_compare}. The mean absolute distance and the standard deviation from the hub centerline to the wake centers for both model are calculated as well. For the mean absolute distance, the results are 15.1m (low-pass filter) and 15.3m (random walk model); for the standard deviation, the results are 19.21m (low-pass filter) and 19.54m (random walk model). All of the results above show that random walk model is able to mimic the low-pass filter in terms of the meandered wake center series.   

\begin{figure}
  \centering
  \includegraphics[scale=0.8]{wake_center}
  \caption{The meandered wake center locations of 100 seconds in the lateral direction.}\label{fig:wake_center_compare}
\end{figure}

The downwind turbine is under the meandered wake of the upwind turbine, therefore the power of the downwind turbine is affected by the behavior of the wake meandering. Thus to determine whether the random walk model is capable of simulating the downwind turbine performance by modeling the meandered wake centers, the power of the downwind turbine of the two models are also compared. 

The power comparison results are shown in Figure \ref{fig:rw_power_compare}. It can be seen that the power of the downwind turbine from two different models are very close. The mean rotor power of the low-pass filter model and the random walk model are 905.17 and 908.10 kW respectively.

\begin{figure}
  \centering
  \includegraphics[scale=0.8]{power2ndTurbine}
  \caption{The power comparison of the downwind turbine by two models.}\label{fig:rw_power_compare}
\end{figure}

All of the comparison results above show that when modeling the meandered wake center, the statistical model random walk is able to mimic the behavior of the physical low-pass filter model and produce accurate power results for the downwind turbine.

\section{Time complexity}\label{sec:timeComplexity}
For a simulation tool, the time complexity is always a very crucial factor to consider when designing the tool. For example, DWM-FAST is a medium fidelity model that could simulate a 10-min turbine operation in minutes.

DWM module consists of two sub-modules - a wake deficit sub-module and a wake meandering sub-module. A few optimizations are made to the wake deficit model to accelerate the speed when resolving the velocity field matrix. 

However, the low-pass filter model becomes the bottleneck when reducing the computation time. Because at each time step and at each downstream distance, all the nodal points in a round disk will be processed in order to obtain the local spatial average velocity, causing a heavy Input-Output (I/O) traffic between the memory and the CPU, which severely reduces the program efficiency. Besides the I/O issue, the computation time complexity of this algorithm is $O(n^3)$.

Meanwhile, using the random walk model could substantially reduce the computation complexity and cost. At each time step and each downstream location, only a CDF is calculated, no more wind file data are required to read and the heavy I/O traffic between the memory and the CPU are eliminated. The computation time complexity for this algorithm is $O(n)$.

For instance, for the test cases mentioned in section \ref{RW_detail}, the running time of the DWM model with the random walk model is 4 seconds. While to finish the simulation with the low-pass filter model, it costs 75 seconds. Which is 17 times more compared to the random walk model.

\section{Utilizing the DWM-FAST program}\label{sec:meandering_recom}
Both low-pass filter and random walk model have been implemented into the DWM-FAST program. A flag option is added into the driver program input file for users to specify which model will be run.

The advantage of the low-pass filter is that it is a physical model, it only depends on the input wind file, and could produce an accurate results if the wind file contains adequate large scale turbulence. If the wind file does not contain adequate large scale turbulence, then artificial scaling factors are required to ensure that the produced meandered wake centers has sufficient variation. The main drawback of the low-pass filter is that it is too expensive in terms of computation resource and computational time. 

It is found that TurbSim is not able to produce a wind file that contains sufficient large scale turbulence and only few people have access to an LES sampled wind file. Thus if accurate results are required from the low-pass filter using the user-generated wind file, scaling factors are needed. The optimum scaling factors could make the results have similar statistics compared with the LES results.

On the other hand, the random walk model is a statistical model whose parameters are trained to match the LES results statistics. The two factors are the walk step size the distribution standard deviation. The walk step size largely impact the autocorrelation value. 

The results show that the random walk model is capable of mimicking the behavior of the results of the low-pass filter and yielding accurate results. The random walk model is very cost efficient in both computation resource and computational time, meanwhile it is easy to modify since it is based on a probability distribution.

The training of the random walk model and the low-pass filter scaling factor are both based on the LES results statistics. Thus it is impossible to choose the parameter before knowing the LES results. But it is also noticed that for the random walk model there are relationships between the parameters and the turbine location that the model for downwind turbine has smaller walk step size compared to that of a upwind turbine.

For example, for the test case of Figure \ref{fig:empirical} and \ref{fig:diag_corr}, the training results of the parameter set (walk step size, standard deviation) for the upwind and downwind turbines are (5.34, 267) and (4.2, 292) respectively. The wake centers autocorrelation of the downwind turbine is larger than that of the upwind turbine thus yielding a smaller walk step size. The parameter grid search is performed in a tridiagonal matrix as shown in Figure \ref{fig:rw_two_turbine}, therefore the standard deviation parameter of the downwind turbine is larger than that of the upwind turbine.

The random walk model test case is based on a LES simulation, in which two NREL 5MW turbines with 7D spacing are used, with the wind speed 8m/s and TI 10\%. Thus if similar situations are required to simulate, the aforementioned parameters could be deployed directly.

The training of random walk model is based on the meandered wake center statistics, which further depend on the wind condition and the turbine operating state. Thus in the future, to build a stronger and robust model, a parameter look-up table could be built after training on sufficient amount of LES or field results.

To conclude, for the low-pass filter method, accurate results will be produced if 1) a wind file with sufficient amount of large scale turbulence is applied 2) or trained scaling factors based on the LES results are deployed. For the random-walk method, accurate results will be produced if the model is trained properly based on the LES results. Thus, if one has a wind file that contains sufficient amount of large scale turbulence, the low-pass filter model is preferred. Otherwise, it is recommended that the random walk model should be deployed due to the fact that the low-pass filter is much more expensive.

\chapter{Modeling Dynamic Wake Effect and the transient period time constant}
In the previous work, it has been demonstrated that the current wind farm wake model is capable of accurately re-producing the wake field and capturing the effects of the wakes on the downwind waked turbine, and in the meantime maintains a low computation cost. However in the current single wake formulation, the wake meanders dynamically, but the wake deficit is static - the inlet boundary condition of the wake is a time averaged quantity of the turbine over a period of time, thus the dynamic changes in the operating state of the upwind turbine are not reflected in the current model. Therefore this precludes anyone from examining turbine dynamic control and the impacts of transients by using this current tool.

In this chapter, a moving average window approach is created in the DWM model to model the dynamic wake effects and obtain the transient period time constant based on the Simulator of Wind Farm Applications (SOWFA) simulation results. First, in section \ref{sec:SOWFA_dynamic}, a SOWFA simulation where the pitch angle of the upwind turbine is changed during the simulation will be introduced. Some post-processing work are performed in order to build and create some fundamental tools before processing and modeling the dynamic transient period. In section \ref{sec:SOWFA_transient}, the wake transient period due to the turbine's pitch angle change of the SOWFA simulation will be investigated and determined. Finally, in section \ref{sec:DWM_transient}, a moving average window model is created and built into the DWM program to model the wake transient period. The results of the moving average window will be compared with the results of the SOWFA simulation.

\section{SOWFA simulation}\label{sec:SOWFA_dynamic}
In this section, the results from a high-fidelity tool are processed and investigated. Two 5MW NREL wind turbines in a row are simulated in SOWFA \cite{SOWFA}. SOWFA is a set of computational fluid dynamics (CFD) solvers, boundary conditions, and turbine models based on the OpenFOAM CFD toolbox. It includes a version of the turbine model coupled with FAST. This tool allows users to investigate wind wind turbine and wind plant performance under the full range of atmospheric conditions and in terrain. At some point of the simulation, the pitch angle of the upwind wind is changed, thus the wake's change due to the turbine's operating state change could be investigated. In this section, a SOWFA simulation where the pitch angle of the upwind turbine is changed during the simulation will be introduced. Some post-processing work are performed in order to build and create some fundamental tools before processing and modeling the dynamic transient period of the SOWFA results.

\subsection{The setup of the SOWFA simulation}
In this SOWFA simulation, two turbines aligned with the inflow wind direction in a row is simulated for 1,000 thousands. The NREL 5MW turbine is deployed as the simulated turbine, with fixed rotor speed to avoid the delay in the turbine control system. The freestream inflow wind speed is 8m/s and the turbulence intensity is roughly 10\%. The initial blade pitch angle is 0 degree. In the case 1, the upwind turbine's pitch angle is changed from 0 to 1 degree at the $600^{th}$ second, then furthermore to 2 degrees at the $800^{th}$ second. In the case 2, the pitch angle of the upwind turbine is changed from 0 degree to 3 degrees at the $500^{th}$ second, then back to 0 degree at the $750^{th}$ second. Figure \ref{fig:sowfa_2turbine} is the velocity plot of the SOWFA simulation 2D field from the top angle of a single time instance, where each black line denotes the location of one crossing plane which is perpendicular to the inflow wind direction. The crossing plane wake velocity profile for each downstream location and each time instance will be extracted and sampled in the following sections.

\begin{figure}
  \centering
  \includegraphics[scale=0.8]{sowfa_2turbine}
  \caption{Velocity plot of the SOWFA simulation 2D field.}\label{fig:sowfa_2turbine}
\end{figure}


\subsection{Smooth the wake profile at a downstream location}
It is useful to identify the wake profile properties at a certain downstream location or compare the wake profiles of two different downstream locations. But the turbulent wake profile makes these tasks very difficult.

To better identify the turbulent wakes, it would be very helpful to smooth the wake profile curve that eliminates some excessive fluctuations. Several smoothing algorithms are tested, among which the $3^{rd}$ order Savitzky-Golay filter \cite{Savitzky} and the robust version of local regression (rloess) filter \cite{Cleveland} work better. Savitzky-Golay filter is a generalized moving average model with filter coefficients determined by an unweighted linear least-squares regression and a polynomial model. Loess uses using weighted linear least squares and a $2^{nd}$ degree polynomial model. 

Figure \ref{fig:smooth_1} and Figure \ref{fig:smooth_2} are the comparisons between the raw wake profile, Savitzky-Golay smoothed wake profile, and rloess smoothed wake profile of the $300^{th}$ second at downstream \textit{300} meters and \textit{500} meters respectively. The y axis denotes the magnitude of the wake axial velocity. The x axis denotes the perpendicular cross-plane index, where 150 corresponds to the rotor hub centerline along the axial direction, 300 and 0 correspond to two diameter to the right and to the left of the hub center line respectively.

\begin{figure}
  \centering
  \includegraphics[scale=0.95]{smooth_1}
  \caption{Smoothed Wake profiles at 300 meters downstream.}\label{fig:smooth_1}
\end{figure}

\begin{figure}
  \centering
  \includegraphics[scale=0.95]{smooth_2}
  \caption{Smoothed Wake profiles at 500 meters downstream.}\label{fig:smooth_2}
\end{figure}

From the results it can be seen that among the two filters, Savitzky-Golay is thought to be the better one. It reflects the width and the depth of the nadir, while smoothing out the turbulence at the wake edge. 

Wake profiles are expected to have different magnitudes at different downstream locations. The whole field is simulated for 1,000 thousands, therefore the time averaged wake velocities at different downstream locations are calculated and compared as Fig  \ref{fig:wake_time_average} shows. The wake profiles at each downstream location are smoothed using the Savitzky-Golay filter. 

\begin{figure}
  \centering
  \includegraphics[scale=0.7]{wake_time_average}
  \caption{Time-averaged wake velocities at different downstream locations.}\label{fig:wake_time_average}
\end{figure}

The time averaged wake profile is less turbulent than the one at a single time instance. When comparing the time average wake profiles between different downstream locations, the slightly difference can be noticed at the wake center - there is a larger minimum wake velocity for the farther downstream location. However, for the 850 m wake profile, the velocity is actually reduced compared with the 750 m wake profile, which is due to the existence of the downwind turbine (the turbine spacing is 880 m). As a result of this, the wake velocity farther than 750 meters might not be usable.

\subsection{Wake advection speed and spatial average wake velocity}
Up to now, we have looked into the time averaged wake velocities and the smoothed local wake profiles, they help identify the different quantities of the wakes at different downstream locations and different time instances. However to deal with the wake released from the turbine plane when there is a change in the operating state, it is crucial to know where the released wake reaches at a certain time or when the released wake reaches a certain location. With these abilities the wake due to the turbine operating state change could be distinguished between the steady state wake before the turbine's operating state change, thus the dynamic wake effect and the transient period time constant could be investigated.

One approach to identify the time scales of the wake’s change due to the changes in the rotor operating condition is to utilize the spatial average velocity of different time instances. Figure \ref{fig:wakeSpeedBefore} present the comparisons of the time series of the spatial average wake velocities between 2 pairs of downstream locations - 450 m and 500 m, 500 m and 550 m. The wake profiles are turbulent as well, but it appears that the wake profiles are uniformly shifted by some time constant $\tau$ - the time the wake spends to travel between the two locations.

\begin{figure}
  \centering
  \includegraphics[scale=0.9]{wakeSpeedBefore}
  \caption{Spatial average velocity time series at different downstream locations.}\label{fig:wakeSpeedBefore}
\end{figure}

To determine the value of the constant $\tau$, the correlation coefficients $R$ of the wake profiles between two neighboring locations with different delay time \textit{t} are computed. For example, $R$ is calculated between the wake profiles at 500 m and 450 m, and between 550 m and 500 m. $R$ represents the similarity of the two wake profiles. The higher the similarity, the larger the correlation coefficients are. The delay time where the maximum $R$ is obtained corresponds to the wake advection time between the two locations. 

Figure \ref{fig:tau} presents the correlation coefficients as a function of the shifted time \textit{t}. It can be seen that for each pair (50 m spacing), the highest value of the correlation coefficients are obtained when the shifted time equal to 8, which means the wake advection time between the two downstream locations is 8 seconds. Thus the large scale wake movement speed is roughly 6.25 m/s. Considering the ambient wind speed is 8 m/s, the normalized wake advection speed is 0.78 $U_{amb}$.

\begin{figure}
  \centering
  \includegraphics[scale=0.8]{tau}
  \caption{Correlation coefficients as a function of delay time \textit{t}.}\label{fig:tau}
\end{figure}

By using the obtained wake advection time, the spatial averaged wake velocity time series are shifted accordingly. Similar to Figure \ref{fig:wakeSpeedBefore}, the spatial averaged velocity time series comparison between two neighboring downstream locations are shown in Figure \ref{fig:wakeSpeedAfter}. The x axis corresponds to the wake releasing time.

\begin{figure}
  \centering
  \includegraphics[scale=0.63]{wakeSpeedAfter}
  \caption{Shifted spatial average velocity time series at different downstream locations.}\label{fig:wakeSpeedAfter}
\end{figure}

After the lateral shift by the time constant $\tau$, the downstream wake profile closely follows the upstream wake profile. Using this technique, it becomes possible to track a certain wake released from the rotor plane through the entire wake field. 

However, there are still some discrepancies in the difference of the spatial average wake velocities time series between different downstream locations due to the turbulence. One test case is shown in Figure \ref{fig:turbulent_wake} - the difference of the spatial average velocities time series between downstream 550 m and 500 m. Unlike what is expected that the values would all be positive since wake velocity is recovered more at a farther downstream location, due to the turbulence, some of the values actually are negative. The wake's nature of turbulence makes the task of wake profile identification challenging.

\begin{figure}
  \centering
  \includegraphics[scale=0.8]{turbulent_wake}
  \caption{The difference of the spatial average wake profile time series between 550 m and 500 m.}\label{fig:turbulent_wake}
\end{figure}

The aforementioned spatial average velocity is calculated by applying a 300 m (2.38D) disk filter whose center is the instantaneous local wake center. Discrepancies in the difference of the spatial average velocity time series between different downstream locations are seen. It would be useful to compare the performance if different sizes of disk filter are applied, thus to determine the appropriate filter size. 

The mean spatial-average velocity difference of each downstream location pair of different filter size is shown in Figure \ref{fig:wake_pair}, where the values corresponds to the the quantities of downstream wake minus the upstream wake. Two averaging filters with different diameters are applied - 300 m (2.38 D) and 200 m (1.58 D). It can be seen that larger filter produces smaller values compared with smaller filter does, it is because larger filter includes more wake outliers, whose velocities are not sensitive to the wake formulation and recovery.

\begin{figure}
  \centering
  \includegraphics[scale=0.8]{wake_pair}
  \caption{Mean spatial average velocity difference of each downstream location pair.}\label{fig:wake_pair}
\end{figure}


\section{Wake transient period investigation in SOWFA}\label{sec:SOWFA_transient}
In this section, first the wake's change due to turbine's pitch angle change will be investigated. Then the time constant of the wake transient period will be determined.

\subsection{Wake's change due to turbine's pitch angle change}
Two cases are simulated in the SOWFA. For the case 1, the pitch angle of the upwind turbine is changed from 0 to 1 degree at the $600^{th}$ second, then furthermore to 2 degrees at the $800^{th}$ second; for the case 2, the pitch angle of the upwind turbine is changed from 0 to 3 degree at the $500^{th}$ second, then back to 0 degree at the $750^{th}$ second. 

The time series of the power production for both turbines in the 2 cases are shown in Figure \ref{fig:power_both_turbine_case1} and \ref{fig:power_both_turbine_case2} respectively. For the case 1, when the pitch angle of turbine 1 is changed at the $600^{th}$ second, the power of turbine 1 is expected to reduce. However, it is not clearly seen in the results. A power spike is noticed at turbine 2 around the $700^{th}$ second, but it then drops to where it was. 

\begin{figure}
  \centering
  \includegraphics[scale=0.8]{power_both_turbine_case1}
  \caption{Power production of both turbines of the SOWFA case 1 simulation.}\label{fig:power_both_turbine_case1}
\end{figure}

\begin{figure}
  \centering
  \includegraphics[scale=0.8]{power_both_turbine_case2}
  \caption{Power production of both turbines of the SOWFA case 2 simulation.}\label{fig:power_both_turbine_case2}
\end{figure}

The pattern is easily seen for the case 2. For the first time the pitch angle is changed (from 0 to 3 degrees), the power of turbine 1 is reduced at the $500^{th}$ second, and the power of turbine 2 is increased around the $640^{th}$ second, followed by a reduction. For the second time the pitch angle is changed (from 3 to 0 degrees), the power of turbine 1 is increased at the $750^{th}$ second. The power of turbine 2 is reduced around the $920^{th}$ second, and followed by a power production increase. In both cases, it is found that the power changes at turbine 2 (either increment or reduction) due to the pitch angle change at turbine 1, are immediately followed by a reverse-bounce. One degree's pitch angle change is insufficient to dominate the wake's formulation over the effect of the inflow turbulence.

Before the pitch angle's change, the wake can be assumed to be in a steady state. Then sometime after the pitch angle is changed, the wake is transformed to another steady state. Thus between the two steady state, it is the wake transient period with respect to the pitch angle change, which is our interest.

To investigate the wake changes with respect to the pitch angle change, instead of looking at the wake velocity as a function of simulation time, which has difficulties discriminating the changes between different locations and different time instances as we have seen before, the wake velocity at each downstream location is investigated as a function of the ”original wake releasing time from the rotor plane”. The spatial average wake is considered to have a large length scale, thus by applying the frozen hypothesis, the wake is able to be tracked from being released from the rotor plane to marching to downstream locations.

By applying this approach, compared with the approach that investigates the wake velocity as a function of the simulation time, the wake response due to the turbine operating state change is separated from the local discrepancy, which is the difference of the inflow turbulence at different time instances. The mapping from simulation time to releasing time at any downstream location is obtained via the wake advection time, which is achieved previous by applying the autocorrelation function.

Basically, for each wake released at the rotor plane at a certain time $t_1$, the wake advection time to a certain downstream location \textit{x} is calculated as $t_2$, thus the wake properties at the downstream location \textit{x} at time $t_1 + t_2$ correspond to the wake released at time $t_1$. To investigate the wake transient period due to the pitch angle’s change, the wake which is released around the time instance when the pitch angle is changed will be focused, and those released wake are tracked while they are marching downstream.

Figure \ref{fig:les_wake_same_release_wake} shows the normalized spatial average wake velocity at 5 different downstream locations around the $600^{th}$ second of releasing time when the pitch angle is changed from 0 to 1 degree. The normalized spatial average velocity is obtained by dividing the absolute spatial average velocity of the downstream location by the one that is at the rotor plane when this wake is released. 

\begin{figure}
  \centering
  \includegraphics[scale=0.80]{les_wake_same_release_wake}
  \caption{Normalized spatial average wake velocity for various downstream locations vs. wake releasing time.}\label{fig:les_wake_same_release_wake}
\end{figure}

The trends of the spatial average velocity are very similar for different downstream locations, which makes sense since they are originated from the same released wake, and this released wake impacts all of the downstream locations. A wake increment is observed from around the $600^{th}$ second, but since the wake profiles are very turbulent, it is hard to justify that these are the wake changes due to the pitch angle change. For instance, at the $660^{th}$ second, a spike is observed but there is no pitch angle change. Therefore a comparison data set is required where everything is identical compared with the current set, except there is no pitch angle change. Using the baseline case, the wake's change due to the pitch angle change can be isolated from the turbulence.

Therefore, a baseline case where there are no pitch angle changes is simulated using SOWFA. The absolute spatial average velocity comparison at 420 m downstream between the simulation with and without pitch angle change is presented in Figure \ref{fig:gaussianfit_abs_released_wake_420m}. It can be seen that before the pitch angle is changed, the two simulation cases are identical. The discrepancy starts from the moment when the pitch angle is changed. The next task is to determine the length of the transient period, which is the window between the $500^{th}$ second and another time instance (\textgreater500) where a new steady state begins. 

\begin{figure}
  \centering
  \includegraphics[scale=0.83]{gaussianfit_abs_released_wake_420m}
  \caption{Absolute average wake comparison with and without pitch angle change.}\label{fig:gaussianfit_abs_released_wake_420m}
\end{figure}

\subsection{Identify the transient period in SOWFA simulation}
In this section, the wake transient time period of LES simulation will be determined statistically. 

Before the pitch angle's change, the wake can be assumed to be in a steady state. Then sometime after the pitch angle is changed, the wake is transformed to another steady state. Thus between the two steady state, it is the wake transient period with respect to the pitch angle change, which is our interest. The steady state before and after the transient period are noted as $ss_1$ and $ss_2$ respectively. In the current case, the pitch angle of the upwind turbine is increased from 0 to 3 degrees at the $500^{th}$ second.

Before post-processing the wake's response to the pitch angle change, it is necessary to explore the response time of the turbine bending moments when pitching the blade in order to check whether there is a delay. The blade out-of-plane bending moments time series near the moment when the blade is pitched is shown in Figure \ref{fig:OOP_bending_pitch_2}. It can be seen that the whole process takes shorter than 1 second thus could be negligible. 

\begin{figure}
  \centering
  \includegraphics[scale=0.83]{OOP_bending_pitch_2}
  \caption{Blade OOP bending moments vs. pitch angle.}\label{fig:OOP_bending_pitch_2}
\end{figure}
 

The quantity $U_{dif}$ in the wake profile to be investigated is defined in Eq. \ref{eq:uDif}. Where $U_{w}$ represents the spatial average velocity of the case where the blade is pitched, and $U_{w/o}$ denotes the spatial average velocity of the case where the blade would not be pitched (i.e. the baseline case). 

\begin{equation}\label{eq:uDif}
  U_{dif} = U_{w} - U_{w/o}
\end{equation}

From the definition it is clear that during $ss_1$, before pitching the blade, $U_{dif}$ should be equal to 0. After pitching the blade, it is expected that $U_{dif}(t)$ slowly increase to a steady level and  should be positive during $ss_2$. From previous results, it is also shown that since the inflow wind is turbulent, during the transient period the behavior is irregular and it is not guaranteed that $U_{dif}$ would be larger than 0. $U_{dif}$ time series at two different downstream locations are presented in Figure \ref{fig:wake_velocity_difference}, which proves the definition.

\begin{figure}
  \centering
  \includegraphics[scale=0.80]{wake_velocity_difference}
  \caption{$U_{dif}$ time series for two downstream locations.}\label{fig:wake_velocity_difference}
\end{figure}

Based on these findings, to find out the transient period length, which is the length of the window between two steady states, the approaches are listed as follows.

\begin{enumerate}
  \item For each downstream location, find the first moment when the wake of pitch blade just reaches the location, which is called the wake arrival time. It is also the end of the steady state $ss_1$, as noted by $t_1$.
  \item For each downstream location, find the first moment when the wake reaches the steady state $ss_2$, as noted by $t_2$.
  \item The transient period length, as noted by $\tau$, is defined by the difference of $t_1$ and $t_2$ as shown in Eq. \ref{eq:transient}.
\end{enumerate}

\begin{equation}\label{eq:transient}
  \tau = t_2 - t_1
\end{equation}

To determine the wake arrival time $t_1$, due to the fact that the wake profiles of $U_{dif}$ are very turbulent after the blade is pitched, to eliminate the effect of the noise, a cutoff speed approach is applied to determine the wake arrival time for a downstream location.

For $U_{dif}$ time series at a certain downstream location $x$, the local wake arrival time $t_1(x)$ is defined as the first moment when $U_{dif}$ reaches the cutoff speed. The cutoff speed is chosen to be very small, but larger than the noise. Typical values are 0.010, 0.015, or 0.02 m/s. The wake arrival time at each downstream location are calculated based on the chosen cutoff speed values.

The wake arrival time results correspond to the three different cutoff speed values are shown in Figure \ref{fig:curoff_0_010}, \ref{fig:curoff_0_015}, and \ref{fig:curoff_0_020} respectively. Besides the wake arrival time results which are noted by the blue lines, a $1^{st}$ order line is fitted based on the wake arrival time results, also two lines with slope value 1/8 and 1/6.8 are also plotted in the same figure as a baseline comparison to show how fast the wake advects.

\begin{figure}
  \centering
  \includegraphics[scale=0.88]{curoff_0_010}
  \caption{Wake arrival time with cutoff speed 0.010 m/s.}\label{fig:curoff_0_010}
\end{figure}

\begin{figure}
  \centering
  \includegraphics[scale=0.88]{curoff_0_015}
  \caption{Wake arrival time with cutoff speed 0.015 m/s.}\label{fig:curoff_0_015}
\end{figure}

\begin{figure}
  \centering
  \includegraphics[scale=0.88]{curoff_0_020}
  \caption{Wake arrival time with cutoff speed 0.020 m/s.}\label{fig:curoff_0_020}
\end{figure}

A few findings from the results.
\begin{itemize}
  \item A linear relationship is found in the wake arrival time and downstream distance.
  \item When the cutoff speed is chosen as 0.010 m/s, lots of noise are noticed in the farther downstream locations, which indicates that the value of 0.010 is not large enough to eliminate the turbulence noise in the wake profile.
  \item No matter what value the cutoff speed is chosen, the trend of the wake arrival time results are very close between each other.
  \item The slope of the fitted $1^{st}$ order line is close to the slope of the 1/6.8 line. Which further proves the results of the wake advection speed investigation where an autocorrelation approach is used.
  \item The fitted $1^{st}$ order line could be used as the wake arrival time for different downstream locations.
\end{itemize}


To determine the first moment $t_2$ when the wake reaches the steady state $ss_2$, due to the turbulence and the nature of a steady state, $t_2$ is defined as the first moment when the wake spatial average velocity reaches the average velocity after the pitch angle’s change.

The transient period length $\tau$ due to the change of blade pitch angle then can be calculated as the difference of $t_2$ and $t_1$, and is shown in Figure \ref{fig:Time_difference_t_reach}. Though there are some noise in the farther downstream locations, but for the most majority of the downstream field, the wake transient period length could be regarded as about 30 seconds.

\begin{figure}
  \centering
  \includegraphics[scale=0.88]{Time_difference_t_reach}
  \caption{Transient period length $\tau$ for a 3-degree pitch angle change.}\label{fig:Time_difference_t_reach}
\end{figure}

The transient period length of 30 seconds corresponds to a 3-degree pitch angle change. To build a general model, and to determine the relationship between the amount of pitch angle change and the transient period length, another case where the blade is pitched by 1 degree is simulated in SOWFA, and the results are post-processed by using the same approaches developed for the 3-degree case.

\begin{figure}
  \centering
  \includegraphics[scale=0.75]{wake_transient_period}
  \caption{Transient period length $\tau$ for a 1-degree pitch angle change.}\label{fig:wake_transient_period}
\end{figure}

The transient period length with different downstream locations for a 1-degree pitch angle change are shown in Figure \ref{fig:wake_transient_period}. The results are not smooth flat, which might due to the reason that 1 degree's change is not sufficient to dominate the turbulence in the inflow wind. The average value of the transient period length is about 10 seconds.



\section{Model the transient period in DWM}\label{sec:DWM_transient}

\subsection{Model the transient period}
In the previous work, it has been demonstrated that the current wind farm wake model is capable of accurately re-producing the wake field and capturing the effects of the wakes on the downwind waked turbine, and maintains a low computation cost. Thus this model may be used to investigate how the downwind turbines react with respect to the dynamic changes in the upwind turbine’s operating state in real time. A similar tool that could capture the dynamic change in the wake does not exist for now. However in the current single wake formulation, the wake meanders dynamically, but the wake deficit is static - the inlet boundary condition is a time averaged quantity over a period of time, thus the dynamic changes in the operation of the upwind turbine are not reflected in the current model. This precludes anyone from examining dynamic control and the impacts of transients by using this tool. Thus DWM should be modified and improved accordingly to address this issue.

When the operating state of a turbine is altered, such as the blade is pitched, for a certain downstream location, the total time $T$ of the dynamic unsteady wake development could be discretized into two time scale terms - the wake arrival time $t_1$ and the transient period length $\tau$ as Eq. \ref{eq:dwm_tau} shows. 

\begin{equation}\label{eq:dwm_tau}
  T = t_1 + \tau
\end{equation}

The wake arrival time $t_1$ defines how long it takes for the wake released from the turbine plane to reach the downstream location. The transient period length $\tau$ is the time that it takes for the wake at a certain downstream location to adjust to the boundary condition’s change: from the time that the flow just starts to react to the boundary condition’s change (the end of $ss_1$) to the time that the flow just reaches the next corresponding equilibrium state (i.e. $ss_2$).

To determine $t_1$ for each downstream location, a Genetic Programming (GP) model created and validated before is deployed. This GP model estimates the wake advection speed thus the wake advection time. Using this model, the time it takes for the wake to travel from the rotor plane to a certain downstream location is able to be calculated. The steps of calculating the wake arrival time for each downstream location are listed as follows.

\begin{enumerate}
  \item Using the GP model, at each downstream location, calculate the local wake advection speed.
  \item Use the calculated local wake advection speed, determine the wake advection time from the current location to the next nearest location. For instance, calculate the wake advection time from 300 m downstream to 310 m downstream.
  \item Repeat steps 1 and 2 for the entire wake field.
\end{enumerate}

When the wake is in the transient period, the wake profile gradually changes from steady state $ss_1$ to $ss_2$. With current DWM wake deficit formulation, the inlet boundary condition is a time averaged quantity over the whole simulation time, thus the dynamic changes in the operation of the upwind turbine are not reflected in the wake formulation. Therefore, the wake deficit model needs to be modified.

To address this issue, a moving average window will be used to calculate the moving average of the rotor boundary condition. This dynamic rotor boundary condition then can be used to update the entire wake field with respect to time. The key of this approach is to maintain a very slow change of the boundary condition such that the wake profiles of two adjacent time steps can be assumed to be semi-steady. There are two major advantages of using the moving average approach. First, the sudden rapid change at the boundary is filtered out, since a sharp boundary condition change may not be handled by the current wake model. Second, the gradual change of the wake inlet at the upwind turbine is able to be captured, such as the change due to the alteration of the pitch angle. Consequently this change at the boundary will be reflected in the unsteady downstream wake development and formulation. 

Certainly, the length of the moving average window will affect the transient period wake formulation. If the size of the window is too large, which means the difference between two adjacent boundary conditions is very tiny, then the corresponding changes in the wake development are too slow; if the size of the window is too small, then the corresponding changes in the wake are too quick.

In this work, to develop this model and validate the results against the LES transient period results, the LES transient period length 30 seconds will be applied as the moving average window length. 

After applying this moving average window and determining the total time $T$, the spatial average velocity time series of two downstream locations are shown in Figure \ref{fig:DWM_spatial_U}, where the pitch angle of the turbine is changed at the $500^{th}$ second from 0 to 3 degrees. It can be seen that due to the wake recovery, the wake velocity at 400m downstream is larger than that of the 300m downstream. The proposed model captures the gradual change of the wake due to the pitch angle's change.

\begin{figure}
  \centering
  \includegraphics[scale=0.88]{DWM_spatial_U}
  \caption{Spatial average velocity time series of two downstream locations.}\label{fig:DWM_spatial_U}
\end{figure}

\subsection{Model validation and comparison}

To validate the transient period results of the DWM moving average window model, the results of the DWM moving average window should be compared with the LES results. In this comparison case, the blade pitch angle is changed at the $500^{th}$ second from 0 to 3 degree.  

The spatial average velocity time series comparison between DWM and LES for two different downstream locations are shown in Figure \ref{fig:transient_compare_300} and \ref{fig:transient_compare_400} respectively. The blue line represents the LES results, the red solid line represents the DWM transient period obtained by using the moving average window model, and the red dashed line shows the speed of the two steady states.
 
\begin{figure}
  \centering
  \includegraphics[scale=0.88]{transient_compare_300}
  \caption{Spatial average velocity time series comparison at 300 m downstream.}\label{fig:transient_compare_300}
\end{figure}

\begin{figure}
  \centering
  \includegraphics[scale=0.88]{transient_compare_400}
  \caption{Spatial average velocity time series comparison at 400 m downstream.}\label{fig:transient_compare_400}
\end{figure}

In general, a good agreement between the LES and DWM results is seen from the figures. The DWM transient period connects the two steady states and the wake velocity gradually transforms from steady state $ss_1$ to $ss_2$. At the DWM transient period location, a wake velocity increment is also seen in the LES results. Next, statistical methods are applied to validate the correctness of the DWM modeled transient period.

The turning points of the DWM wake profile correspond to the intersection between two neighboring states. When the wake profile is fixed, the location of the turning points determine the quantities and properties of the two steady states and the transient period. Thus in order to check if the three states are correctly modeled in the DWM moving average model, one could determine whether the location of those intersection points are correct or appropriate, and whether the wake velocity in all the three states matches the results of LES simulation. Therefore two subjects will be investigated:

\begin{enumerate}
  \item During the steady state $ss_1$, steady state $ss_2$, and the transient period, whether the DWM wake profile reflects the LES mean wake velocity?
  \item What is the relationship between the wake speed of the transient period and the other two steady states?
\end{enumerate}


Therefore, the mean LES velocities over three states are calculated respectively as a function of downstream distance, and are compared with the DWM results. The comparison results for $ss_1$, transient period, and $ss_2$ are shown in Figure \ref{fig:speed_staeady_state_1_compare}, \ref{fig:speed_transient_compare}, and \ref{fig:speed_staeady_state_2_compare} respectively.

\begin{figure}
  \centering
  \includegraphics[scale=0.88]{speed_staeady_state_1_compare}
  \caption{Normalized mean velocity comparison in steady state $ss_1$.}\label{fig:speed_staeady_state_1_compare}
\end{figure}

\begin{figure}
  \centering
  \includegraphics[scale=0.88]{speed_transient_compare}
  \caption{Normalized mean velocity comparison in the transient period.}\label{fig:speed_transient_compare}
\end{figure}

\begin{figure}
  \centering
  \includegraphics[scale=0.88]{speed_staeady_state_2_compare}
  \caption{Normalized mean velocity comparison in steady state $ss_2$.}\label{fig:speed_staeady_state_2_compare}
\end{figure}

Two conclusions can be drawn from the results.
\begin{enumerate}
  \item In all three states, especially for the transient period, the wake formulation of DWM matches the wake velocity of the LES simulation.  
  \item The transient period velocity is larger than that of the steady state $ss_1$, and smaller than that of the steady state $ss_2$, which indicates that the transient period captures the transformation between the steady state $ss_1$ and $ss_2$.
\end{enumerate}

To conclude, the validation and comparison results show that the velocities of all three states are successfully modeled in the DWM model by using the moving average window approach. The modeled transient period of DWM matches that of the LES simulation in terms of length and magnitude. 

Therefore the modified DWM model is able to capture the dynamic change in the wake with respect to the dynamic changes in the upwind turbine's operating state in real time. In the future, this tool could be used as the tool to examine dynamic control and the impacts of transients.

\chapter{Modeling the wakes of offshore floating turbines}
In the previous chapters, it has be demonstrated that WFMP is capable of modeling the wakes of an onshore fixed platform turbine and the dynamic changes in the wake transient period. The combination of WFMP and FAST 8 enables several other analysis, such as how the wake is produced by an offshore floating wind turbine, which were not able to be performed previously.

In this chapter, the approaches of modeling the dynamic wakes of offshore floating turbines using WFMP will be developed and discussed.

\section{Offshore wind power}
Offshore wind power refers to the construction of wind farms in bodies of water to generate electricity from wind. Offshore wind is very popular in recent years and is expanding rapidly. The offshore wind power have advantages and disadvantages over their onshore counterparts. The wind resource offshore consists of both higher wind speeds and lower turbulence, as well as lower wind shear. The higher wind speed allows for offshore wind turbine capacity factors to be higher on average, and the lower turbulence and shear reduce wind-induced blade loads and increase the wind turbine lifespan. In addition, visual and noise impacts are reduced due to the larger distance from residences.

Offshore wind resources are abundant, stronger, and blow more consistently than land-based wind resources. Data on the technical resource potential suggest more than 4,000 gigawatts (GW) could be accessed in state and federal waters along the coasts of the United States and the Great Lakes \cite{Schwartz}.

Figure \ref{fig:US_wind} provides the wind resources estimated at a 80 m height for all 50 states—the 48 contiguous states, Alaska, and Hawaii—as well as offshore resources up to 50 nautical miles from shore, where the superior wind resource available offshore is shown. Also, due to the fact that most of the population of the United States are near the coasts, the power produced by offshore wind can be used near the load centers, reducing transmission costs. 

\begin{figure}
  \centering
  \includegraphics[scale=0.4]{US_wind}
  \caption{Wind resource map of the United States \cite{Schwartz}}\label{fig:US_wind}
\end{figure}

However, the drawbacks of offshore wind power require attention. The major drawback is the increased support structure costs as well as the increased cost of maintenance. In addition, offshore turbines experience increased loading from waves and currents, which necessitates stronger and more expensive components.

Figure \ref{fig:offshore_wind} shows how the design of the offshore wind turbine must change to economically access deeper water. Most installed offshore wind turbines today use a monopile type foundation, which consists of a single large tube driven into the seabed as the support structure. However, since much of the offshore wind resource in the United States and throughout the world is located over deep water, floating platforms are now being developed and prototyped to access this resource.

\begin{figure}
  \centering
  \includegraphics[scale=0.5]{offshore_wind}
  \caption{Natural progression of substructure designs from shallow to deep water \cite{Jonkman}}\label{fig:offshore_wind}
\end{figure}

Figure \ref{fig:offshore_platform} presents the typical three designs of offshore floating wind turbines, which are classified by the stabilizing method. Those designs are not completely distinct, since each includes some component of the other stability mechanisms. For example, the spar buoy and barge-style platforms both require some stability from mooring lines, but the dominant stability mechanism is ballast and buoyancy, respectively.  

\begin{figure}
  \centering
  \includegraphics[scale=0.5]{offshore_platform}
  \caption{Floating platform concepts for offshore wind turbines  \cite{Jonkman}}\label{fig:offshore_platform}
\end{figure}


\section{5MW NREL turbine with OC3 spar buoy platform}\label{sec:spar}
Originally a floating spar buoy design is developed by a Norwegian company to support a Siemens 2.3 MW turbine in the Hywind project. NREL modified this design to be compatible with the NREL 5MW turbine, and the result is called the OC3-Hywind Spar Buoy \cite{oc3}.

The OC3 design is shown in Figure \ref{fig:oc3}. A heavy counterbalance at the base of the spar is applied to lower the center of mass. This creates a restoring moment if the spar is pitched or rolled. When the spar is yawed, it is balanced by the ``Y'' shaped mooring line. The offshore code comparison collaboration (OC3) developed corrections to the model to account for this mooring configuration as well as other model discrepancies that were discovered with the use of experimental comparisons.   

\begin{figure}
  \centering
  \includegraphics[scale=0.5]{oc3}
  \caption{Illustrations of the NREL 5-MW wind turbine on the OC3-Hywind spar buoy \cite{oc3}}\label{fig:oc3}
\end{figure}

In terms of modeling the wake, the difference of modeling the wakes of an onshore wind turbine and an offshore floating wind turbine is coming from the motion of the floating platform. The spar buoy motion are as shown in Figure \ref{fig:oc3_motion}. The motions include three translational components (heavy in the vertical direction, sway in the lateral direction, and surge in the axial direction) and three rotational components (yaw about the vertical axis, pitch about the lateral axis, and roll about the axial axis). The motion of the platform depends on the inflow wind and wave condition.

\begin{figure}
  \centering
  \includegraphics[scale=0.7]{oc3_motion}
  \caption{Degrees of freedom for an offshore floating wind turbine platform \cite{Tran}}\label{fig:oc3_motion}
\end{figure}


\section{Wake modeling of floating wind turbines}
The test turbine whose wake will be modeled using WFMP in this work is chosen as the NREL 5MW turbine sitting on the OC3 spar buoy as described in the section \ref{sec:spar}.

The challenges of modeling the wakes of a floating turbine come from the motion of the platform itself. At each moment, the wake released from the turbine will have different centerline locations and different axial angles, thus generates various impacts on the downwind turbine.

For the motions of a spar buoy floating platform. Sway, surge, and heave are within a range of few meters, compared with the several hundreds meters of turbine spacing between the upwind and downwind turbines, the wake effect due to those three motions on the downwind turbine could be regarded as negligible. The most significant and unique motion of the spar buoy is the pitch, which mainly depends on the wave and the inflow wind condition. For a upwind turbine, the pitch will cause the wake to be redirected upwards, which may cause a weaker weak on the downwind turbines. 

The motion of the spar buoy due to the wind and wave are modeled by the FAST V8. More specifically, the HydroDyn \cite{hydrodyn} and MAP++ \cite{map} modules. HydroDyn is a time-domain hydrodynamics module that has been coupled into the FAST wind turbine computer-aided engineering (CAE) tool to enable aero-hydro-servo-elastic simulation of offshore wind turbines. It is used to model the hydrodynamic loads on the turbine. The Mooring Analysis Program (MAP++) is a library designed to be used in parallel with other CAE tools to model the steady-state forces on a Multi-Segmented, Quasi-Static (MSQS) mooring line, thus the response of the offshore platform could be modeled.

The platform pitch angle time series with different wind speed values are shown in Figure \ref{fig:Ptfm_pitch_angle_ws_1} and \ref{fig:Ptfm_pitch_angle_ws_2}. In this case, a wave with 6 meter significant wave height and $0^\circ$ wave direction is applied. 

There are two findings from this result. First, larger wind speed values result in larger pitch angles. Second, the platform pitch angle is keeping changing due to the fact that the wind and wave are keeping changing as well.
 
\begin{figure}
  \centering
  \includegraphics[scale=0.9]{Ptfm_pitch_angle_ws_1}
  \caption{Platform pitch angle for various wind speed values}\label{fig:Ptfm_pitch_angle_ws_1}
\end{figure}

\begin{figure}
  \centering
  \includegraphics[scale=0.9]{Ptfm_pitch_angle_ws_2}
  \caption{Platform pitch angle for various wind speed values}\label{fig:Ptfm_pitch_angle_ws_2}
\end{figure}

To model the dynamic effect of the platform pitch motion on the wake formulation and the downwind turbine power/load. The meandered wake center locations are modeled dynamically.

The approaches of modeling the dynamic meandered wake center locations of floating wind turbines are inspired by the ``low-pass'' filter model and the approaches of modeling the wake trajectory of a yawed turbine described in section \ref{sec:wake_meandering} and \ref{sec:turbine_yaw} respectively. 

In the low pass filter model, Taylor's frozen turbulence hypothesis is applied for the downstream advection of the wake. Adopting Taylor's hypothesis enables the separation of the wake-releasing of the rotor plane of each different time steps. In addition, the properties of the wake field at a certain crossing plane in the meandering frame of reference do not change throughout the whole process.

To re-direct the wake trajectory of a yawed turbine, the skew angle is modeled based on the findings by Jimenez et al. using momentum conservation theory \cite{Jiménez} as expressed in Eq. \ref{eq:alpha} and Eq. \ref{eq:dy} respectively. Accumulating those small wake center displacements at each time step, the entire re-directed wake center locations could be modeled. 

In this case, platform pitch could be regarded as similar to the turbine yawing, except platform pitch causes the wake to be redirected upwards or downwards. In contrast, yawing causes the wake to be redirected leftwards or rightwards.

Based on and developed from the two models, the steps of modeling the dynamic wake trajectory of a floating wind turbine are listed as follows.
\begin{enumerate}
  \item Using FAST 8, the platform pitch angle time series is obtained. The platform pitch angle is the key dependency of redirecting the wake of a spar buoy platform.
  \item In the wake meandering cascade model, at time $t_i$, a wake $\omega_i$ will be released from the rotor plane. The next wake $\omega_{i+1}$ is released at  $t_i+ \beta$ seconds.
  \item The value of $\beta$ is very tiny, thus a time average platform pitch angle $\sigma_i$ could be used to quantify the overall platform pitch motion between $t_i$ and $t_i+ \beta$. $\sigma_i$ is calculated for each window between $t_i$ and $t_i+ \beta$. Therefore each wake $\omega_i$ released in step 2 from the rotor plane has a corresponding pitch angle $\sigma_i$. 
  \item When releasing each wake $\omega_i$ from the rotor plane, using the obtained pitch angle $\sigma_i$ to re-direct the wake trajectory vertically.
  \item While wake $\omega_i$ is marching downstream, continue re-directing the wake trajectory using its original pitch angle $\sigma_i$ obtained when it was originally released from the rotor plane.
  \item Repeat steps 2 to 5 for each wake released from the rotor plane.
  \item Superimpose this wake on the downstream turbine.  
\end{enumerate}

Figure \ref{fig:OC3_vertical_wake_center} shows the meandered wake center vertical locations at different time steps with and without modeling the dynamic wake re-direction caused by the OC3 platform pitch motion, where ``Improved model'' means it dynamically re-directs the wake trajectory using the platform pitch motion. It can be seen that the wake is re-directed upwards due to the platform pitch motion.   

\begin{figure}
  \centering
  \includegraphics[scale=0.9]{OC3_vertical_wake_center}
  \caption{Meandered wake center vertical locations at various time moments}\label{fig:OC3_vertical_wake_center}
\end{figure}

By superimposing this dynamic wake onto the downwind turbine, the corresponding power and loads could be calculated. The time series of the turbine power and blade out-of-plane bending-moment (OOP-BM) are shown in Figure \ref{fig:power_comparison_fst8} and \ref{fig:bending_moment_comparison_fst8} respectively.   

\begin{figure}
  \centering
  \includegraphics[scale=0.9]{power_comparison_fst8}
  \caption{Time series of the downwind turbine power}\label{fig:power_comparison_fst8}
\end{figure}


\begin{figure}
  \centering
  \includegraphics[scale=0.9]{bending_moment_comparison_fst8}
  \caption{Time series of the downwind turbine BR-OOP bending moment}\label{fig:bending_moment_comparison_fst8}
\end{figure}


It could be seen that the downwind turbine yields a higher power production when the upwind turbine pitch motion is included. This is because the wake is dynamically re-directed upwards by the platform pitch motion, thus the downwind turbine experiences an inflow wind with higher mean wind speed. The normalized power of the downwind turbine with and without including the pitch motion modeling are \texttt{0.4665} and \texttt{0.4492} respectively.

For the turbine loading, the mean blade OOP-BM of the downwind turbines for the two models are close, the main discrepancy is the bending moment variation. The model which includes the modeling of the upwind pitch motion yields a higher standard deviation: \texttt{3988.3 kN*m} compared with \texttt{3946.7 kN*m}. This result is expected because the blade flies in and out of the partial waking when the wake is re-directed upwards, thus creating larger loads fluctuations.

In conclusion, WFMP is proved to be capable of modeling the dynamic wakes of offshore floating platform wind turbines. To enable the dynamic wake simulation of offshore floating turbines, the wake meandering model is improved to include the dynamic effect that the wake is released with different pitch angles at different times, therefore the wake is dynamically redirected accordingly. Then the corresponding effects are reflected on the downwind turbines and captured by WFMP. 



\chapter{Conclusions}
Turbine wakes are of primary interest when designing, modeling, and controlling a wind farm. However, turbine wakes in a wind farm represent a significant engineering challenge. This dissertation has developed and presented a number of novel solution approaches associate with these problems. As large wind farms and floating offshore wind turbines become more common around the world, the conclusions of this dissertation will be more widely used. 

The three important questions this research has answered are:
\begin{enumerate}
  \item How to systematically, efficiently, and accurately model the wakes of a wind farm with arbitrary turbine layout and arbitrary inflow condition?
  \item How to efficiently and accurately model the meandered wake center locations using a flexible approach?
  \item How to identify and model the dynamic wake changes of both onshore and floating wind turbines using an efficient approach that could be further developed for control needs?  
\end{enumerate}

This chapter will outline the primary conclusions presented by this study and will also list potential avenues for future investigations.

\section{Summation of Primary Conclusions}  
The broad objective of this dissertation is to provide tools and approaches which can model the wakes of an arbitrary wind farm with arbitrary inflow condition, and further improves the wake modeling to enable the examination of the wind turbine dynamic control in a wind farm. In addition, since offshore floating wind turbines are the future of deepwater wind energy, this developed model is further proved to be capable of modeling the dynamic wakes of floating wind turbines. 

Conclusions and contributions regarding the development of the Wind Farm Modeling Program (WFMP) (Chapter 4-6):
\begin{itemize}
  \item WFMP consists of a single wake model (DWM model) and a computer aided simulator capable of simulating horizontal-axis wind turbines. The single wake model consists of a wake deficit sub-module and a wake meandering sub-module. WFMP is able to obtain the wake boundary condition at the rotor plane, resolve the wake velocity, model the meandered wake center locations, and superimpose the modeled wake onto the downwind turbine to estimate the turbine power and loads accordingly.
  \item The performance of WFMP is validated and compared against the high-fidelity LES results and the field data in terms of turbine power and loads. The results show WFMP is capable of modeling the wakes for a wind farm with arbitrary turbine layout and arbitrary inflow condition.
  \item By applying momentum conservation theory, wake redirecting of a yawed turbine is included in WFMP. Therefore the downwind turbine power and loads are estimated accordingly.
  \item With the incorporation of FAST, WFMP provides a unified, flexible, and efficient approach for wind farm efficiency estimation and turbine loads assessment. A input text file is provide to users to specify customized wind farm parameters. WFMP has been released by NREL and can be downloaded at \textit{https://nwtc.nrel.gov/DWM}.     
\end{itemize}

Conclusions and contributions regarding the development of the random walk model (Chapter 8):
\begin{itemize}
  \item By reviewing the drawbacks of traditional meandered wake center models, a statistical approach based on Markov Chain and random walk is developed, trained, and validated.
  \item The random walk model has been proved to be able to mimic the behavior of the physical low-pass filter model and produce accurate power results for the downwind turbine.
  \item There are two advantages of the random walk model compared with the low-pass filter model. First, it is cost-efficient and costs much less time compared with the low-pass filter model. Second, random walk model is flexible because it is easy to modify and can be applied to any case as long as the probability distribution is known. 
  \item Both random walk model and low-pass filter model are implemented in the WFMP. A flag option is added into the driver program input file for users to specify which model will be run.
\end{itemize}

Conclusions and contributions regarding the development of modeling dynamic wake effects (Chapter 9-10):
\begin{itemize}
  \item The dynamic wake changes in the transient period due to the upwind turbine's operating state change is investigated both statistically and experimentally.
  \item A moving average window model is developed and created in the WFMP to model the wake dynamic changes in the transient period. The results of the moving average window model are validated and compared against the high fidelity LES results. This developed tool could help one examine turbine dynamic control.
  \item The combination of WFMP and FAST 8 enables several other analysis, such as how the wake is produced by an offshore floating wind turbine, which were not able to be performed previously. In order to model the dynamic wakes of offshore floating turbines, the wakes of floating platform wind turbines are dynamically modeled in WFMP, and the corresponding dynamic effects are captured on the downwind turbines. 
\end{itemize}

\section{Recommendations of future investigation}
When simulating the wakes and their effects on the downwind waked turbines, the results shown in this work present the challenges of designing an apples-to-apples comparison between the field data of a large wind farm and simulation tools. Both of the WFMP and LES simulation tools are found to underestimate the turbine relative power compared with the field data. This is due to the uncertainty in the wind direction and the wind speed. For instance, for a large offshore wind farm, the wind direction of a upwind turbine and a further downstream turbine might not be the same, it is possible that a downwind waked turbine experiences a freestream from other directions other than the upwind turbine direction. These uncertainties are not modeled by simulation tools currently. Two different paths could be taken to address this issue. First, install more meteorological masts in the wind farm to quantify the wind plant inflow with higher solution and more detail, thus to reveal the unknown wind uncertainty. Second, instead of modeling the wakes for a fixed condition, the models could be brought in some factors related to random wind directions and wake recovery rate, thus to match the field data statistically. 

Fifty percent or more time spent in running WFMP is occupied by reading the inflow wind files. The wind file reading is repeated for each wind turbine even if an identical wind file is used. This inefficient and repeated wind file reading could be eliminated by introducing a wrapper program, from which the wind data of each turbine are inherited from. Also the turbines are simulated sequentially, this calculation may be parallelized. The resulting parallelization would yield significant speed improvements. Cluster computing services, like Amazon’s Elastic Compute Cloud (EC2), provide a cheap and convenient option for parallelization. Speed up of the program is a prerequisite for wind farm layout optimization and large wind farm simulations. 

The simulation of a waked turbine's performance heavily rely on the modeling of meandered wake center locations. In this work, an efficient and versatile approach is developed. To further develop this tool, more field data and LES results are required to reveal and model the relationship between the wind condition and the behavior of wake center meandering. 

The wake's dynamic change has a significant impact on the downwind turbine's control. Alternative approaches to addressing the dynamic wake change should be investigated. This includes releasing the instantaneous wake which reflects the instant turbine operating state change and marching it downstream. A fairly efficient time complexity should be guaranteed by this solution for the turbine control needs.

The wakes of offshore floating turbine are modeled by including the platform pitch motion. The combination of WFMP and FAST in terms of the offshore wind application enables several other analysis which are not able to be performed previous, including, but not limited to, the mooring dynamics analysis and the hydro-elastic analysis of waked offshore wind turbines.
  
%%%%%%%%%%%%%%%%%%%%%%%%%%%%%%%%%%%%%%%%%%%%%%%%%%%%%%%%%%%%%%%%%%%%%%%%%%%%%%%%%%%%%%%%%%%%%%%%%%%%%%%%%%%%%%%%%%%%%%%%%%%%%%%%%%
%% End of body
%%%%%%%%%%%%%%%%%%%%%%%%%%%%%%%%%%%%%%%%%%%%%%%%%%%%%%%%%%%%%%%%%%%%%%%%%%%%%%%%%%%%%%%%%%%%%%%%%%%%%%%%%%%%%%%%%%%%%%%%%%%%%%%%%%

\appendix
\chapter{Inputs of DWM driver program}
In this chapter, detailed explanation and usage caution of the DWM driver program inputs are discussed.\\

\noindent HubHt: The turbine hub height [m]

This is the turbine hub height, the value of this parameter cannot be negative.\\

\noindent RotorR: The radius of rotor blade [m]

This parameter specifies the distance from the rotor apex to the blade tip. The value of this parameter cannot be negative.\\

\noindent NumWt: The total number of wind turbines [-]

This parameter specifies the total number of the wind turbine that to be simulated. The value of this parameter cannot be smaller than 1.\\

\noindent Uambient: The ambient wind velocity [m/s]

This is the mean wind speed at the turbine hub height. The value of this parameter cannot be negative.\\

\noindent TI: The ambient turbulence intensity [\%]

This is the average ambient turbulence intensity in percentage at the hub height or the TI of the TurbSim generated wind file. The value of this parameter cannot be negative.\\

\noindent ppR: The number of points per radius [-]

DWM uses the finite difference scheme to solve the fluid dynamic system. This parameter defines the number of calculation node in the spanwise direction. In general, larger node numbers cost more calculation time but yield more precise results. The value of this parameter must be an integer and 50 is recommended.\\

\noindent Domain\_R: The radial domain size [R]

DWM applies an axisymmetric coordinate system to solve the fluid dynamic system. This parameter specifies the radial distance from the center of the calculation domain to the very boundary edge and is scaled by the rotor radius. The value of this parameter cannot be smaller than 3 and should be adjusted based on the turbine layout if there are more than single turbines in this simulation. If the wind direction is not down the row, then the radial domain size should be larger than spacing*sin(theta) where theta is the turbine alignment angle.\\

\noindent Domain\_X: The longitudinal domain size [R]

This is the longitudinal domain size counted from the investigated rotor plane. The value of this parameter must be positive and should be determined based on the turbine layout if there is more than a single turbine in this simulation. For example, this value should be larger than the maximum spacing between two neighboring turbines if a downstream turbine is only affected by the closest upwind turbine.\\

\noindent Meandering\_simulation\_time\_length: The total number of simulation time steps in the meandering wake model for a fixed cross plane [-]

\noindent Meandering\_Moving\_time: The total number of simulation time steps in the meandering wake model for a moving cross plane [-]

These two parameters control the wake meandering sub-model. Taylor’s frozen turbulence hypothesis is applied for the downstream advection of the wake, and the fundamental assumption of this approach is that the wake transport in the atmospheric boundary layer can be modeled, by considering the wake to act as a passive tracer driven by large scale turbulence. To calculate the wake displacement in the vertical and lateral directions, the wake is modeled as constituted by a cascade of wake deficits, each “emitted” at consecutive equally spaced time increments, in agreement with the passive tracer analogy.

At every time step, a cross-plane is released from the rotor plane and marches downstream while the wind properties of this cross-plane are constant. The entire space domain behind the turbine consists of multiple cross-planes, which are all released from the rotor plane with the spacing between two neighboring cross-planes equal to the ambient velocity multiplied by the time step interval. In each cross-plane there is a wake center position. For a single cross-plane, it marches a constant distance downstream at every time step, and a new wake center position is obtained using a filter function, with the averaged vertical and lateral velocity calculated based on the wake center position of the last time step.

As shown in Figure \ref{fig:wake_meandering}, first, at time $t_0$, a frozen cross-plane $C_0$ is released from the turbine plane $A_0$ and starts to advect downstream. After time $\triangle t$, this frozen cross-plane $C_0$ arrived at the location of the cross plane $A_1$. Finally at time $T$ this frozen cross plane $C_0$ reaches the location where the cross-plane $A_n$ locates. Throughout this whole process at each time step and at each cross plane location $A$, a meandered wake center vertical and lateral position coordinate in the frozen cross-plane $C_0$ is returned by applying a low-pass filter, and the wake center coordinate at the x direction is equal to the ambient velocity multiplied by the local total time which is counted from the original release at the rotor plane. A frozen cross-plane $C$ is released at every time step consecutively and the same approach discussed above is applied for every subsequent frozen cross-plane.

The parameter Meandering\_Moving\_time specifies the total time T for which a frozen cross plane travels, and the parameter Meandering\_simulation\_time\_length specifies the total number of the cross planes that advect from the turbine cross plane $A_0$ to the very downstream boundary plane $A_n$.

The values of the two parameters both must be positive integers. And to obtain the appropriate parameters values, Eq. \ref{eq:MSTL} and Eq. \ref{eq:MMT} are shown below.
\begin{equation}\label{eq:MSTL}
  MSTl\geq Ztime/(\frac{\frac{2R}{ppR}}{0.32U_{ambient}}\cdot 10)+1
\end{equation}
\begin{equation}\label{eq:MMT}
  MMT\geq spacing\cdot\frac{ppR}{10}+1
\end{equation}
\\

\noindent WFLowerBd: The lower bound of wind file [m]

This parameter is used only when a TurbSim generated wind file is read in as the wind input. The value of this parameter is equal to the value of the TurbSim wind file lower bound when generating the wind file.\\

\noindent Winddir: The ambient wind direction [$ ^\circ $]

This parameter specifies the incoming ambient wind direction in units of degrees. The value of this parameter must be between $0^\circ$ and $360^\circ$. The degree measurements are top-wise as shown in Figure \ref{fig:wind_direction}. If looking down on the wind farm, 0$^\circ$ means the wind comes from the very top or north, 90$^\circ$ the very right or east, 180$^\circ$ the very bottom or south and 270$^\circ$ the very left or west.
\begin{figure}
  \centering
  \includegraphics[scale=0.6]{wind_direction}
  \caption{Illustration of wind direction degree measurements}\label{fig:wind_direction}
\end{figure}
\\

\noindent RanW: A flag indicating whether or not to use random walk (-)

This parameter specifies whether or not to use random walk to model the meandered wake center locations, where 1 stands for true and 0 stands for false. If false is detected by the program, low-pass filter model will be applied to model the meandered wake center locations.\\

\noindent XCoordinate, YCoordinate: The coordinate of turbine [D]

These parameters are the coordinates of the turbines at x and y directions in the Cartesian coordinate system, and they are scaled by the rotor diameter. The total number of the lines must be equal to the total number of the turbines. At each line, the x coordinate is typed in first and it is followed by the y coordinate of the same turbine. Based on the order of the turbine coordinate inputs, the turbines are sorted numerically from 1 to the total number of turbines.



%%
%% Beginning of back matter
\backmatter  %% <--- mandatory

%%
%% We don't support endnotes

%%
%% A bibliography is required.
\interlinepenalty=10000  % prevent split bibliography entries
\bibliographystyle{umthesis}
%\bibliographystyle{unsrtnat}
\bibliography{Yujia_bib}
\end{document}

%%% Local Variables:
%%% mode: latex
%%% TeX-master: t
%%% End: 